%%
%% This is file `thesis.tex',
%% generated with the docstrip utility.
%%
%% The original source files were:
%%
%% nudtpaper.dtx  (with options: `thesis')
%% 
%% This is a generated file.
%% 
%% Copyright (C) 2018 by Liu Benyuan <liubenyuan@gmail.com>
%% 
%% This file may be distributed and/or modified under the
%% conditions of the LaTeX Project Public License, either version 1.3a
%% of this license or (at your option) any later version.
%% The latest version of this license is in:
%% 
%% http://www.latex-project.org/lppl.txt
%% 
%% and version 1.3a or later is part of all distributions of LaTeX
%% version 2004/10/01 or later.
%% 
%% To produce the documentation run the original source files ending with `.dtx'
%% through LaTeX.
%% 
%% Any Suggestions : LiuBenYuan <liubenyuan@gmail.com>
%% Thanks Xue Ruini <xueruini@gmail.com> for the thuthesis class!
%% Thanks sofoot for the original NUDT paper class!
%% 
%1. 规范硕士导言
% \documentclass[master,ttf]{nudtpaper}
%2. 规范博士导言
% \documentclass[doctor,twoside,ttf]{nudtpaper}
%3. 建议使用OTF字体获得较好的页面显示效果
%   OTF字体从网上获得,各个系统名称统一。
%   如果你下载的是最新的(1201)OTF英文字体,建议修改nudtpaper.cls,使用
%   Times New Roman PS Std
% \documentclass[doctor,twoside,otf]{nudtpaper}
%   另外,新版的论文模板提供了方正字体选项FZ,效果也不错哦
% \documentclass[doctor,twoside,fz]{nudtpaper}
%4. 如果想生成盲评,传递anon即可,仍需修改个人成果部分
% \documentclass[master,otf,anon]{nudtpaper}
%
\documentclass[master,otf,twoside,anon]{nudtpaper}
\usepackage{mynudt}

\classification{TP391}
\serialno{16063284}
\confidentiality{公开}
\UDC{004.8}
\title{分布式深度学习系统中高效参数\\通信技术的研究与实现}
\displaytitle{分布式深度学习系统中高效参数通信技术的研究与实现}
\author{陈晓涛}
\zhdate{\zhtoday}
\entitle{Research and Implementation of Efficient Parameter Communication Technology in Distributed Deep Learning System}
\enauthor{Chen Xiaotao}
\endate{\entoday}
\subject{集成电路专业}
\ensubject{Integrated Circuits}
\researchfield{分布式深度学习}
\supervisor{李东升\quad{}研究员}
\cosupervisor{张钊宁\quad{}助理研究员} % 没有就空着
\ensupervisor{Prof. Li Dongsheng}
\encosupervisor{Assistant Prof. Zhang Zhaoning} % 没有就空着
\papertype{专业学位}
\enpapertype{Engineering}
% 加入makenomenclature命令可用nomencl制作符号列表。

\begin{document}
\graphicspath{{figures/}}
% 制作封面,生成目录,插入摘要,插入符号列表 \\
% 默认符号列表使用denotation.tex,如果要使用nomencl \\
% 需要注释掉denotation,并取消下面两个命令的注释。 \\
% cleardoublepage% \\
% printnomenclature% \\
\maketitle
\frontmatter
\tableofcontents
\listoftables
\listoffigures

\midmatter
\begin{cabstract}
近年来以神经网络为基础的人工智能技术在学术界和工业界得到了广泛应用和发展。随着神经网络模型和训练所需的数据量不断增加,使得单机训练神经网络越来越困难。分布式训练神经网络不仅可以极大减少训练时间,也可完成某些单机情况下无法训练的神经网络。在可预见的未来,分布式训练神经网络将成为必然选择。如何提高分布式训练神经网络的效率则显得尤为重要。

本文针对这一问题,在提出低精度分布式更新LPDU算法,将原始浮点数梯度转换为BF16格式进行传输,减少同步时间开销,进而提升分布式训练效率,通过混合精度更新算法,保证训练精度。本文通过分析LPDU算法各部分的时间开销和神经网络的参数规模得出LPDU算法在特定神经网络训练中的理论性能提升,并通过实验加以验证。通过对比LPDU算法与原始算法在图像分类,目标检测任务的相关实验证明:LPDU算法在图像分类与目标检测任务中均能达到与原始更新算法相同的理想精度。同时对分布式训练性能有所提升。在8节点情况下,resnet50的效率由原始的84.05\%提升到了87.50\%,VGG网络的训练效率相对于原始的79.42\%提升至了86.55\%,SSD网络相对于原始效率有4.83\%的提升。

基于LPDU算法减少梯度尾数位的思路,本文提出两种极限精度梯度压缩EPGC方法:9比特梯度压缩方法和8比特梯度压缩方法。9比特梯度压缩方法是在浮点数基础上去除所有尾数位,仅使用1个符号位和8个指数位表示梯度;8比特梯度压缩方法是在半精度浮点数基础上去除8位尾数位,使用1个符号位,5个指数位和2个尾数位表示梯度。为快速验证本文提出的梯度压缩方法的可行性,本文通过在原始浮点数或半精度浮点数基础上对特定尾数位置零的方式模拟这两种压缩方法。通过实验证明这两种梯度压缩算法均能保证神经网络在图像分类任务中的训练精度,说明通过这两种梯度压缩算法提升分布式训练效率上可行的。

\end{cabstract}
\ckeywords{神经网络; 分布式训练; 低精度;更新算法; 梯度压缩}

\begin{eabstract}
In recent years, artificial intelligence technology based on neural networks has been widely used and developed in academia and industry. As neural network models and the amount of data required for training continue to increase, it becomes increasingly difficult to train neural networks in a single machine. The distributed training neural network not only can greatly shorten the training time, but also can train some neural networks that cannot be trained in single machine. In the foreseeable future, distributed training neural networks will become an inevitable choice. How to improve the efficiency of distributed training neural networks is particularly important.

This paper proposes a low-precision distributed update (LPDU) algorithm, which converts the original floating-point gradient into BF16 format for transmission, which reduces the synchronization time overhead and improves the distributed training efficiency. The Mixed precision update  (MPU) algorithm can ensure the training accuracy. In this paper, by analyzing the time overhead in each part of LPDU algorithm and the parameter size of neural network, the theoretical performance improvement of BF16 distributed update algorithm in specific neural network training is obtained and verified by experiments. By comparing the training accuracy curves between the LPDU algorithm and the original algorithm, it proves that the LPDU algorithm can achieve the same ideal precision as the original update algorithm in both image classification and object detection tasks. And the performance of distributed training has improved. In the case of 8 nodes training, the efficiency of resnet50 has increased from the original 84.05 \% to 87.50\%. The training efficiency of the VGG has increased to 86.55\% compared with the original 79.42\%, and the SSD network has increased by 4.83\% compared with the original efficiency.

Based on the idea of LPDU algorithm to reduce the mantissa of gradient, two extreme precision gradient compression (EPGC) algorithms are proposed: 9-bits gradient compression algorithm and 8-bits gradient compression algorithm. The 9-bits gradient compression algorithm removes all mantissa bits in the float data format, using only 1 sign bit and 8 exponent bits to represent the gradient; the 8-bits gradient compression algorithm removes 8 bits in the mantissa of half-precision data. using one sign bit, five exponent bits and two mantissa bits represent the gradient. In order to quickly verify the feasibility of the two proposed gradient compression algorithms proposed, this paper simulates the two compression algorithms by zeroing the specific mantissa position based on the original floating point number or half-precision floating point number. Experiments show that both gradient compression algorithms can guarantee the training accuracy of neural networks in image classification tasks. It is feasible to improve the efficiency of distributed training through these two gradient compression algorithms.

\end{eabstract}
\ekeywords{neural network; distributed training; low precision; update algorithm; gradient compression}



%书写正文,可以根据需要增添章节; 正文还包括致谢,参考文献与成果
\mainmatter
\chapter{绪论}
本章首先阐述了本课题的研究背景及现实意义,然后介绍国内外针对该问题的研究与进展。同时说明在实际应用中存在的问题和可进一步优化的方法。最后介绍本论文的整体组织结构。
\section{研究背景与意义}
近年来,以神经网络\upcite{alexnet2012}为代表的深度学习方法在图像处理、机器视觉、自然语言处理等诸多应用领域取得了巨大突破。自2012年AlexNet\upcite{alexnet2012}以绝对优势夺得ImageNet冠军后,神经网络逐渐得到学术界、工业界的广泛关注,开启了人工智能新篇章。随后各种性能更优、结构更复杂的网络如雨后春笋般层出不穷。如VGG、inception系列、resnet系列,denseNet等\upcite{vgg2014, inception2015, resnet2016, denselynet2017}。同时,鉴于卷积神经网络在图像分类领域的优异表现,业界将其应用到了目标检测、图像分割等领域。如fast rcnn系列\upcite{rcnn2014, fastrcnn2015, fasterrcnn2015}、yolo系列\upcite{yolo2016}、FPN\upcite{fpn2017}、R-FCN\upcite{rfcn2016}等。在自然语言处理领域,涌现出了一系列以循环神经网络[引用]为基础的深度学习方法,在机器翻译、文本分析、阅读理解等\upcite{gnmt2016, bert2018}问题上取得了显著进步,达到甚至超越人类水平。

因为神经网络参数量、计算量巨大,往往需要花费数天,乃至更长的时间。如最初的AlexNet\upcite{alexnet2012},需要5-6天训完。使得网络更新迭代设计周期变长,极大限制了科研人员的研究进度。随着网络结构复杂化迭代周期过长的问题尤为严重。分布式训练神经网络则能极大缩短训练周期,加速研究迭代、产品孵化等。同时,随着神经网络的不断发展以及社会对人工智能应用需求的提升,深度学习所需的数据量、网络模型也越来越大,单台计算机算力已经很难甚至无法满足其计算要求。为了减少神经网络的训练时间和适应不断增长的算力要求,分布式训练神经网络是必然选择。

本课题围绕着分布式训练神经网络的效率和精度进行展开,旨在保证网络精度前提下,提升分布式训练神经网络效率和分布式训练系统的可扩展性。本课题不仅能解决现今训练神经网络时间过长的问题, 更为今后单机场景下无法完成训练的神经网络提供了高效的训练方法。这对于现阶段和可预见未来都有重要实用价值,促进神经网络的发展,对人工智能的发展有深远意义。

\section{研究现状}
随着人工智能的快速发展,为提升训练效率,满足生产需求。业界已针对分布式深度学习系统进行了深入的研究。目前业界主流深度学习框架同时支持parameter servers和horovod进行分布式训练。如tensorflow,pytorch,mxnet等\upcite{tensorflow2016, torch2002, mxnet2015}均支持pamameter servers和horovod。在通信效率上,针对神经网络的训练特点,CMU邢波教授课题组提出了一系列经典有效的方法:WFBP,SFB等。其petuum系统也受到广泛关注。在训练精度上,业界针对大batch size训练提出了一系列更新算法,保证大batch size的训练精度。如facebook何恺明等\upcite{train1hour2017}就提出线性增大学习率的方法保证了在batch size达到8k时,和单机batch size等于256时一致的精度,使得大规模分布式训练神经网络成为可能。随后,伯克利尤洋等提出LARS算法\upcite{train24min2017},在保证精度的情况下,进一步将batch size扩大到32k;紧接着Google Brain提出动态调整batch size的方法\upcite{dontdecay2018}将batch size提升到64k。
\subsection{分布式通信框架介绍}
目前主流深度学习框架均支持parameter servers和horovod。parameter servers最早由smola教授提出,最终由其学生李沐实现\upcite{ps2014}。其提供同步、异步、半异步的更新方式。因其灵活的更新方式和良好的性能得到业界广泛认可。参数服务器通过key-value键值对来管理参数。通常情况下,参数服务器是以去中心化的形式分散在多个节点,各个节点负责部分数据。其通信架构如图~\ref{fig:ps_comm_pattern}所示。所有的server共同维护整个网络的参数,server之间可以相互通信,client端将模型参数根据key分别传给相应管理这部分参数的server。Client之间不存在通信。

\begin{figure}[htp]
\centering
\includegraphics[width=10cm]{ps_comm_pattern}
\caption{参数服务器通信架构图}
\label{fig:ps_comm_pattern}
\end{figure}
参数服务器主要有5大优点:(1)通信高效,在异步情况下不会拖慢计算;(2)支持弹性一致性,其可通过放宽模型一致性条件,达到算法收敛速度和系统性能之间的平衡。(3) 扩展性强,增加节点时无需重启网络;(4)机器错误恢复时间短,Vector Clock容许网络错误了;(5)易用性,其全局共享的参数使用向量和矩阵表示,而这些又可以用高性能多线程库进行优化。

参数服务器下,训练流程如图~\ref{fig:ps_update_way}所示。训练主要包含4部分:(1)worker节点各自计算梯度;(2)worker节点将各自梯度push到server端;(3)server端对所有梯度进行求和平均并更新模型参数w;(4)worker节点从server端pull最新的参数w。

\begin{figure}[htp]
\centering
\includegraphics[width=10cm]{ps_update_way}
\caption{参数服务器下训练流程图}
\label{fig:ps_update_way}
\end{figure}
horovod\upcite{horovod2018}是uber基于百度ring-allredue的通信方式,提出了的分布式通信框架,因为其优异的性能和简单的使用方式,在业界引起广泛关注。horovod基于mpi实现,为提高分布式通信的效率,提出tensor fusion和分层级同步策略等方法,极大提高了分布式训练效率。 为方便用户查找分布式程序中的错误,分析程序性能等,其提供了Timeline监测工具,该工具可跟踪horovod中所有的事件,并以时间轴的形式展示,如图~\ref{fig:hvd_timeline_chrome}所示。极大简化了用户的分析工作,其为用户优化分布式程序性能提供了简单,有效的分析手段。
\begin{figure}[htp]
\centering
\includegraphics[width=10cm]{hvd_timeline_chrome}
\caption{chrome浏览器中timeline时刻表}
\label{fig:hvd_timeline_chrome}
\end{figure}

\subsection{低精度训练神经网络介绍}
因为传统硬件计算基本以浮点数为主,目前绝大多数训练场景中均采用浮点数计算。而神经网络本身存在随机性和容错性的特点,其计算精度并不需要达到浮点数的精度,为了节省内存,带宽,以及在支持低精度计算的硬件设备上提升计算效率,业界提出了两种主流的低精度数据格式用于神经网络训练。分别是IEEE半精度浮点数(FP16)和bfloat16(BF16)格式。两者与浮点数据格式区别如表~\ref{tab:diff_format_bits}所示。

\begin{table}[htbp]
\centering
\begin{minipage}[t]{0.9\linewidth}
\caption{不同数据格式比特位的详细分布}
\label{tab:diff_format_bits}
\begin{tabularx}{\linewidth}{l X X X X}
\toprule[1.5pt]
{\song 数据格式} & {\song 总比特位数} & {\song 符号位数} & {\song 指数位数} & {	\song 尾数位数}\\
\midrule[1pt]
FP32 & 32 & 1 & 8 & 23\\
FP16 & 16 & 1 & 5 & 10\\
BF16 & 16 & 1 & 8 & 7\\
\bottomrule[1.5pt]
\end{tabularx}
\end{minipage}
\end{table}
由科学计数法可知,表~\ref{tab:diff_format_bits}中3种数据格式在不同指数情况下,精度范围有所不同,3种数据格式所能表示的最小精度如表~\ref{tab:diff_format_precision}所示。经过工业界验证,用这两种数据格式训练神经网络均保证训练精度,逐渐成为主流。目前主流的深度学习框架tensorflow,pytorch,mxnet均支持半精度浮点数计算,BF16数据格式目前只有tensorflow针对自家硬件TPU有所支撑。

\begin{table}[htb]
\centering
\noindent\begin{minipage}{0.45\textwidth}
\centering
\caption{不同数据格式最小精度}
\label{tab:diff_format_precision}
\begin{tabular}{p{2cm}p{2cm}}
\toprule[1.5pt]
数据格式 & 数据精度 \\\midrule[1pt]
FP32 & 0.00000012 \\
FP16 & 0.00390625 \\
BF16 & 0.03125000 \\
\midrule[1pt]
\end{tabular}
\end{minipage}
\end{table}
2017年Micikevicius P等为节省训练的内存开销,提出混合精度训练\upcite{mixed2018}的方法,使用IEEE半精度格式存储所有网络层的中间结果。相对于原始32位浮点数计算而言,训练网络模型所消耗的内存减小了一半;同时,为了弥补半精度表示所带来的损失,文章提出混合精度更新方法,如图~\ref{fig:fp16_training_steps}所示。在设计损失函数时,适当放大损失函数权重的方法来保证训练精度。

\begin{figure}[htp]
\centering
\includegraphics[width=10cm]{fp16_training_steps}
\caption{混合精度训练迭代方法}
\label{fig:fp16_training_steps}
\end{figure}
文章也指出,当前大部分硬件不支持半精度计算,所以此方法在现有硬件基础上只能节省内存,由于实际计算是先把半精度数值转换成浮点数再计算,存在一定的额外开销,使得计算速度相对有所下降。目前Nvidia的最新硬件,如Volta系列,turing系列已经支持半精度计算,可进一步提升计算效率。

受硬件条件和框架限制,只有Google公司将BF16数据格式用于训练神经网络,相关研究相对较少。根据其在ML perf\upcite{mlperf_result}中提交的相关结果可知,在图像分类、目标检测、自然语言处理等经典神经网络应用场景中BF16训练均能达到原始浮点数相同精度,证明了BF16数据格式用于神经网络训练的可行性和有效性。

\section{本文工作}
本课题旨在提高神经网络在分布式训练中的效率和分布式系统的可扩展性。根据目前业界对低精度数据在神经网络中的研究现状可知,BF16或FP16格式数据即可保证神经网络的精度。本课题提出BF16分布式更新算法,算法核心思想是把原始浮点数梯度转换为BF16格式数据进行全局同步,使得同步数据量减半,进而减小同步时间开销,提高分布式训练效率。在BF16分布式更新算法基础上,本文通过进一步减少梯度数据的有效尾数位继续减少同步数据量,以提升分布式训练系统的可扩展性和训练效率。结合半精度浮点数的数据格式,本文分别探索出两种梯度压缩方法:9比特梯度压缩方法和8比特梯度压缩方法。实验证明这两种压缩方法均能保证分类网的训练精度,证明这两种压缩算法的可行性。

\subsection{BF16分布式更新算法}
BF16数据格式所需的数据位与半精度数据格式一致。其传输的数据量不变,而其指数位与浮点数一致,精度位比浮点数少了16位。相对于半精度数据而言有两个优势:1.在与浮点数进行转换的过程中,直接将浮点数的低16位舍去即可,无需对数据位进行转化。其与浮点数的转化更为高效;2.因为其指数位长度与浮点数一致,故其表示范围与浮点数基本一致,在进行数据转化时,不存在数值越界的情况。同时,因为BF16指数位相对较长,使得其精度位相对较短,其表示的精度相对半精度数据要差。

分布式训练相对于单机训练方式而言,仅多了同步梯度的过程,其同步开销大小决定了分布式训练效率的高低。考虑到现有硬件的特点和深度学习对低精度数据的友好性,本文提出使用浮点数据计算,BF16数据同步梯度的训练方法。通过减少网络通信量的方法,提高分布式训练效率。本文提出的BF16分布式更新算法主要包含下面三部分,具体设计与实验将在第三章详细介绍:

1. 在进行梯度同步之前,将各层梯度转换成bfloat16数据格式;

2. 基于horovod实现自定义的BF16数据格式的加法,将其传入MPI用于BF16梯度的求和;

3. 在完成BF16数据同步后,将BF16的梯度转换为浮点数表示,用于下一次迭代计算。

\subsection{9比特与8比特梯度压缩方法}
考虑到梯度数据的特殊性:数值往往非常小,绝大部分梯度绝对值在0~1范围内。且随着训练过程中网络逐渐趋向于稳定,梯度数据的波动范围更小,更集中到0~1范围,可使用更少比特位表示梯度。通过减少梯度数据的表示位,可减少分布式训练程中的同步数据量,进而提高分布式系统的可扩展性和训练效率。

根据梯度数据这一特点,本文提出了专门用于针对梯度数据的9比特压缩方法和8比特压缩方法,本文希望进一步减少梯度数据尾数位,探索在保证神经网络训练精度情况下梯度数据所需的最少比特位。为快速验证在特定尾数位情况下神经网络的训练精度,本文通过将原始浮点数或半精度浮点数特定尾数位清零的方法模拟特定比特位梯度压缩方法。经实验验证:在浮点数基础上仅保留2比特、1比特或去除所有尾数位均能保证分类网的训练精度。故本文提出去除所有尾数位的9比特(1个符号位,8个指数位)梯度压缩方法。结合半精度浮点数的特点,本文提出8比特梯度压缩方法:1个符号位,5个指数位,2个尾数位。实验证明该压缩算法同样能保证分类网的训练精度。具体设计与实验将在第四章详细介绍。

\section{本文组织结构}
本文针对现有硬件特点和神经网络对低精度数据的友好性,提出两种低精度数据传输方式,在保证训练精度的前提下,减小数据通信量,提升了分布式训练效率。文章主要探讨了在分布式训练神经网络中低精度数据传输的可行性以及不同数据表示的优劣点。本文主要分为五章,详细安排如下:

第一章,绪论。主要对本课题的研究背景和研究意义进行介绍,系统介绍了目前以神经网络为主的深度学习的发展,分布式深度学习系统的发展和低精度训练神经网络的最新进展以及现阶段计算硬件的限制。并说明了本文的主要研究内容以及对应的创新点。最后说明了本文的组织结构。

第二章,分布式训练神经网络和低精度训练的相关研究。首先介绍了分布式训练神经网络的基本概念,包括数据并行和模型并行两种典型并行方法;然后介绍了近年来神经网络在图像分类、目标检测、自然语言处理等领域的应用于发展;最后介绍了分布式深度学习的发展,包括分布式深度学习系统,优化算法和数据压缩等方面的研究进展。

第三章,BF16分布式更新算法。完整介绍了论文提出的BF16更新算法的实现以及实验结果。首先介绍了将horovod整合进horovod中的实现方法;然后介绍了BF16分布式更新算法的设计与实现;最后具体介绍了相关实验内容,分析验证了BF16更新算法在不影响神经网络训练精度的情况下对分布式训练效率的提升效果。

第四章,神经网络梯度压缩方法探索。分别介绍了两种梯度压缩方法:9比特梯度压缩方法与8比特梯度压缩方法。通过在浮点数或半精度浮点数基础上对特定尾数位置零的方法模拟这两种压缩方法。通过具体实验结果分析说明这两种梯度压缩方法的有效性。

第五章,总结与展望。对本课题的工作做出总结,说明本课题的两个创新点,指出基于目前研究进展未来需要做的下一步工作。

\chapter{分布式深度学习相关研究}
本章主要介绍分布式深度学习系统涉及的相关技术。从分布式深度学习系统的特点出发,阐述其中设计的难点和重点。从精度和效率两方面分析分布式深度学习系统的特点;介绍近年来神经网络在不同领域的发展以及目前业界对分布式深度学习的研究进展。
\section{分布式训练神经网络的特点}
随着网络模型的不断增大,数据日益增加以及应用需求的剧增,在深度学习发展历程中,分布式训练神经网络是必不可少的一环。目前主要有数据并行和模型并行两种方法用于分布式训练,如图2.1所示。\\
\begin{figure}[htp]
\centering
\includegraphics[width=10cm]{data_model_parallel}
\caption{数据并行(左)与模型并行(右)示意图}
\end{figure}
数据并行就是把数据分到不同的计算资源节点上,让不同节点处理不同的数据。因为有多个节点在并行的处理数据,所以在单位时间内,理应能够处理更多的数据,也即达到了并行计算的目的。模型并行则是把一个大的网络模型划分成很多小块,然后把每小块对应的参数、状态和计算任务分在不同的计算节点上执行,再想办法进行同步。模型并行除了通过模型并行获得加速外,当某些模型在单机上无法训练的时候(如内存不够),则只能通过模型并行才能完成训练。目前业界绝大部分训练以数据并行的方式进行训练。\\
分布式训练神经网络主要有两个指标:一是精度,即随着分布式规模不断增大,在数据并行方法下必然导致用于训练的神经网络的batch size线性增大,即每次迭代更新所用的数据增多,而整体的迭代次数线性下降。当batch size增长到接近数据集大小时,sgd算法就不再适用。在大batch size情况下,保证单机batch size一样的精度是分布式训练神经网络中重要难题。尤其是在分布式规模较大情况下,传统sgd更新算法已经无法保证训练精度,需要新的更新算法保证网络精度。如facebook提出的热启动、线性增大学习率方法[引用],在保证训练精度情况下,将batch size提升至8k;伯克利的尤洋等[引用]提出的分层自适应学习率的方法,进一步将batch size提升至32k;以及谷歌大脑[引用]提出通过逐渐增大batch size替换学习率衰减的方法,最终将batch size提升至64k,只用2500多次迭代就将ImageNet训到了理想精度。二是效率,即在保证精度的情况下,训练系统的效率要尽可能高,分布式相对于单节点的加速比越接近理想加速比则性能越好。而根据分布式训练更新特点可知,分布式训练相对于单机训练区别在于多了同步全局梯度的开销,计算开销与单机相同。故同步梯度开销的大小对分布式深度学习系统的效率至关重要。\\
为减少同步梯度的开销,涌现出了一系列异步,半异步的方法,如Hogwild[Hogwild!: A Lock-Free Approach to Parallelizing Stochastic Gradient Descent]提出lock free方法,在稀疏问题上具有显著效果;Sixin Zhang等人提出弹性均值随机梯度下降算法[Deep learning with Elastic Averaging SGD],通过预估节点梯度与全局梯度的差值来弥补异步更新的差异,达到理想效果;Peter等[How to scale distributed deep learning?]提出gossiping sgd算法,类似于去中心化的EASGD算法,在节点规模较小的情况下,收敛速度快于传统sgd算法;随后,Ioannis Mitliagkas等[Asynchrony begets Momentum, with an Application to Deep Learning]提出一种新颖的momentum更新方法,将节点梯度与全局梯度的差距当成隐含的momentum来解释,从而提出通过设置反向momentum的方法抵消局部梯度与全局梯度之间的差异。目前神经网络的训练基本是用sgd算法或sgd算法的变种。\\
在分布式训练神经网络中,异步或半异步的sgd算法无法达到正常收敛水平;若要使用异步半异步的方法现有更新算法要大改,所以目前分布式训练神经网络主要采用严格同步的更新方法,这使得提高训练系统效率更为困难。
\section{神经网络的发展}
classification, detection,natural language process\\
近年来神经网络因其突出的性能在各个领域得到广泛应用。下面分别以分类、目标检测、自然语言处理三个领域对神经网络发展进行介绍。
\subsection{神经网络在图像分类领域的发展}
2012年AlexNet[引用],将卷积神经网络用于图像分类的开山之作。在ImageNet上刷新各项纪录,夺得冠军。继而掀起一股神经网络热潮。其网络结构如图2.2所示。\\
\begin{figure}[htp]
\centering
\includegraphics[width=10cm]{AlexNet}
\caption{AlexNet网络结构示意图}
\end{figure}
受当时计算算力和GPU显存的影响,将模型放在两块GPU上共同训练,总共8层。最终以超过第二名8.2\%的分数夺得冠军。\\
随后,DeepMind提出了性能更强网络结构更深的VGG系列网络。同时相对于AlexNet,该网络结构也更加规则化,为神经网络提供了一个模块化设计的思路,为以后设计更大更深的网络提供思路。此后基于卷积神经网络的各种基本组件,就可以像搭积木一样设计出相应的网络结构。\\
同时,google也提出性能更强的,类似网中网的网络结构Inception系列。即组成网络的基本组件不再是简单的单层结构。而是偏复杂的module结构,类似一个小型的网络模块,如图2.2所示。\\
\begin{figure}[htp]
\centering
\includegraphics[width=10cm]{inception_picture}
\caption{inception module结构示意图}
\end{figure}
如图2.3分别展示了下采样和不下采样情况下,module的结构模型。随后基于inception提出了一系列的改进版如inception-V2,inception-V4等。从中可以看出网络设计越来越趋向于结构化。卷积核的形状也越来越规整,基本以3x3卷积为主,没有了诸多不同尺寸的卷积核设计。\\
随后,Kaiming He等[引用]提出声名大噪的ResNet,一举夺得CVPR 2016年最佳学术论文奖。其计算量,参数量远小于VGG,inception,而精度则比其他网络要高。其核心思想来源于传统机器学习中的残差学习。提出让逐层网络学习上一层网络的残差,比直接学习特征要相对容易。基本残差结构如图2.4所示。
\begin{figure}[htp]
\centering
\includegraphics[width=10cm]{resnet_block}
\caption{ResNet block结构示意图}
\end{figure}
2017年,Gao Huang等[引用]基于残差连接的思想,提出DenseNet。其核心思想是为了使后层网络获取前部分网络尽可能多的信息。使其分别与前面各层建立连接,从而形成相对密集的连接网络,即DensetNet,如图2.5所示。\\
\begin{figure}[htp]
\centering
\includegraphics[width=10cm]{DenseNet_picture}
\caption{DenseNet结构示意图}
\end{figure}
随着越来越多的神经网络部署到实际应用中,其对网络计算量提出了许多限制要求。因为在算力受限的设备中如:手机,个人电脑以及其他嵌入式设备等,无法对计算量大的网络完成计算,或计算量过大则无法满足其实时性要求。故受限算力下的小型网络结构应运而生。\\
2017年,Andrew G. Howard等[引用]提出轻量级网络MobileNet。其核心思想是根据传统卷积的作用:扩大感受野和融合通道间的特征,分别用计算量偏小的depthwise convolution和pointwise convolution代替,极大减少了网络的计算量,其替换原理图如图2.6所示。在相同计算量的情况下,达到了非常好的性能。\\
\begin{figure}[htp]
\centering
\includegraphics[width=10cm]{mobilenet}
\caption{传统conv+BN+Relu(左)与depthwise separable conv(右)示意图}
\end{figure}
随后,Xiangyu Zhang等[引用]提出ShuffleNet,进一步减小了网络计算量。在相同计算量情况下,网络性能达到最好情况。其核心思想是把pointwise convolution进行分组,根据卷积计算公式可知,分组越多,计算量越少,而通道间信息传递就越少。为了使通道间信息能更好的融合,论文提出shuffle的方式将各层通道的信息进行融合,示意图如图2.7所示。其主要两点保证网络性能:一是对pointwise做分组卷积;二是对分组后的卷积在通道之间做shuffle,保证通道间的信息融合。\\
\begin{figure}[htp]
\centering
\includegraphics[width=10cm]{shufflenet}
\caption{bottleneck单元(a)组卷积单元(b)stride=2的组卷积单元(c)示意图}
\end{figure}
\subsection{神经网络在目标检测领域的发展}
随着神经网络在图像分类领域取得的显著进展,人们开始将卷积神经网络用于目标检测领域。近年来卷积神经网络在该领域也取得了突飞猛进的发展。目前基于神经网络的检测算法主要分为两种:一阶段检测网络和二阶段检测网络。一阶段网络速度较快,精度偏低;二阶段网络速度偏慢,精度较高。下面分别介绍一阶段网络和二阶段网络中相关的研究进展。\\
2014年,Ross Girshick等[引用]提出的R-CNN网络是将神经网络用于目标检测领域的开山之作。其主要分为两个步骤:1.用传统方法selective-search从原始图片中选取出2k个框;2. 将这些框对应的图片送入卷积神经网络,让网络对这些框进行分类进而得出对应框及其类别。其算法流程图如图2.8所示。\\
\begin{figure}[htp]
\centering
\includegraphics[width=10cm]{rcnn_overview}
\caption{R-CNN算法流程图}
\end{figure}
由其算法流程图可知,该算法流程耗时较复杂。因为一张图片同时提取2k个框,占用磁盘空间较大,如5000张图片会产生几百G的特征文件,且因为一张图片中的每个特征框都得过一遍神经网络,其处理速度非常慢,平均47秒处理一张图片。\\
针对R-CNN训练速度慢的问题,2015年Kaiming He等[引用]提出SPPNet。其分析了R-CNN之所以速度慢的原因在于一张图片要过2k次神经网络,根本原因在于不同框的尺寸不同,而网络最后的分类部分要求特征图尺寸一致,这种矛盾使得2k个框只能分别过神经网络。为克服这个问题,论文提出空间金字塔池化方法,其结构示意如图2.9所示。通过该方法将不同尺寸的特征图池化到相同尺寸,从而使得一张图片只需过一遍神经网络,极大提升了网络性能。\\
\begin{figure}[htp]
\centering
\includegraphics[width=10cm]{sppnet}
\caption{SPP结构示意图}
\end{figure}
随后,Ross Girshick等[引用]根据R-CNN的缺点,以及SPPNet的优点,提出Fast R-CNN,主要有两个工作:1.将R-CNN中SVM的分类部分用卷积神经网络代替;2.改进SPP的池化方法。Fast R-CNN的网络结构如图2.10所示。\\
\begin{figure}[htp]
\centering
\includegraphics[width=10cm]{fast_rcnn}
\caption{Fast R-CNN算法流程图}
\end{figure}
进一步,基于Fast R-CNN的工作,Shaoqing Ren, Kaiming He和Ross Girshick共同提出Faster R-CNN检测网络。为了解决Fast R-CNN网络提取框耗时的问题,论文提出用于提取检测框的FPN网络,进一步提升了训练速度,实现了真正意义上的端到端训练。FPN网络结构如图2.11所示。\\
\begin{figure}[htp]
\centering
\includegraphics[width=10cm]{fpn}
\caption{FPN网络结构图}
\end{figure}
随后也出现了各种网络如:R-FCN,focal loss[引用]等。\\
另一方面,一阶段目标检测网络也发展迅猛。2016年,Joseph Redmon等[引用]提出yolo网络,其核心思想是直接让网络预测出每个特征点的类别和对应框的4个坐标。其网络结构如图2.12所示。因其检测速度快的优势,得到广泛使用,随后基于yolo不断进行改进,yolo2,yolo3也相继发表提出。\\
\begin{figure}[htp]
\centering
\includegraphics[width=10cm]{yolo}
\caption{yolo网络结构图}
\end{figure}
同年Wei Liu等[引用]提出SSD网络,结合了Faster R-CNN和yolo的优点,采用了Faster R-CNN的anchors机制,同时是和yolo相同的一阶段检测方法。网络结构如图2.13所示。其使用不同层的feature map预测不同的大小的框,不同层的feature map可以预测不同大小的物体,达到预测不同大小物体的目的。\\
\begin{figure}[htp]
\centering
\includegraphics[width=10cm]{ssd}
\caption{ssd网络结构图}
\end{figure}
\subsection{神经网络在自然语言处理领域的发展}
在自然语言处理领域,神经网络被用在了机器翻译,文本处理,阅读理解等诸多方面。2016年,谷歌提出其机器翻译模型GNTMT[引用]。其引入了注意力机制使得系统处理长句子时更准确高效。针对神经网络系统的三个弱点:1.训练速度很慢并且需要巨大的计算资源,由于数量众多的参数,其翻译速度也远低于传统的基于短语的翻译系统;2.对罕见词的处理很无力,而直接复制原词在很多情况下肯定不是一个好的解决方法;3.在处理长句子的时候会有漏翻的现象。GNMT使用了8层含有残差连接的循环神经网络,其结构如图2.14所示。残差连接可以帮助某些信息,比如梯度、位置信息等的传递。同时,attention层与decoder的底层以及encoder的顶层相连接。\\
\begin{figure}[htp]
\centering
\includegraphics[width=10cm]{gnmt}
\caption{gnmt网络结构图}
\end{figure}
2018年,谷歌提出BERT[引用]模型,在11个NLP任务中刷新了成绩,是一项里程碑式的工作。使用了masked lm和下一句预测的联合训练方法,在加上12层的transformer。\\
\section{分布式深度学习的发展}
分布式深度学习涉及到的领域很广。涵盖优化算法、模型、系统和应用等。下面分别介绍以下三个方向的研究进展:分布式系统的发展,优化算法的研究现状和数据压缩的进展。\\
\subsection{分布式深度学习系统的发展}
自2012年底AlexNet[引用]在ImageNet上大放异彩,刷新诸多记录,从此开启了深度学习的新篇章。同年,对应的分布式深度学习系统也应运而生。Jeffrey Dean等提出的DistBelief[19],即Google的第一代深度学习系统,其核心实现是把分布式机器学习里面的数据并行(data parallelism)和模型并行(model parallelism)用在深度学习上,通过这两种并行方法把深度学习大规模化到集群。同时,论文提出了分布式异步SGD—DownPour SGD和L-BFGS算法,在CPU集群上系统效果显著。\\
随后,考虑到分布式训练网络过程中的同步通信开销,Qirong Ho[20]等提出了松弛一致性模型:Stale Synchronous Parallel(SSP), 如图2.15所示。解决了严格同步(BSP)模型下木桶效应导致同步时间过长的问题,提高了计算效率。并基于该模型,搭建了Petuum系统。\\
\begin{figure}[htp]
\centering
\includegraphics[width=10cm]{ssp_picture}
\caption{SSP模型示意图}
\end{figure}
同时,Adam Coates[21]等第一次搭建了一个GPU集群:COTS HPC system,该系统通过模型并行分布式训练网络模型。因为该系统GPU用于计算,CPU用于通信。在InfiniBand 50 Gb网络上测试,计算在毫秒级完成,而通信时间则需要8秒,系统效率极差,为克服通信瓶颈,其提出一种复杂有效的GPU协作算法,进一步了提升集群效率。\\
紧接着Microsoft推出了基于参数服务器的深度学习系统Adam[22]。其是一个CPU集群,通过模型并行的方式训练神经网络。为避免数据处理速度跟不上,系统专门用四台机器用于数据预处理,并将模型放在L3 cache上。为了提升网络间数据的传输性能,该系统设计了一种专门的网络IO,避免不必要的数据拷贝、传输。\\
分布式深度学习中一个关键的通信组件是参数服务器(Parameter Server),最早由Alex Smola提出,因其简单、高效的特点,被广泛用于分布式深度学习进行参数同步。最著名的参数服务器框架要数李沐等人做的Parameter Server[23],并进一步开发出MxNet。在李沐等人的Parameter Servers框架中,其集成了传统分布式框架Hadoop、Spark的优点,通过图遍历、消息压缩和去中心化等优化方法,理论性能与allreduce一致,实际性能十分优异,但其对CPU需求较大,当CPU承担任务过多时,容易因为竞争冲突导致性能下降。\\
随后,Henggang Cui等[引用]在GPU上实现了Parameter Servers,推出GeePS系统,避免了CPU通信过程中GPU与CPU之间的数据拷贝,提高了数据通信效率。同时为缓解模型占用GPU显存过多的问题,提出一种GPU内存管理机制,使得在GPU受限情况下,训练更大的网络成为可能。
\subsection{分布式深度学习中优化算法的发展}
在分布式深度学习系统中主要分为异步算法和同步算法。下面分别介绍分布式深度学习中近年来异步和同步算法的研究进展。\\
2011年,Feng Niu等[引用]针对稀疏求解问题,提出异步更新方法即lock-free方法。因为问题是稀疏的,每个worker所需要更新的参数几乎都不一样,重叠率很低。所以不用严格顺序更新,异步更新即可。2014年,Henggang Cui等[引用]提出一种结合ssp和wpc-bsp的算法A-BSP。算法原理图如图2.16所示。大致内容为:在bsp中每次迭代都同步更新梯度,使得同步等待的时间在总体训练时间过长。为了减少同步等待的时间比重,提出每间隔wpc次迭代才同步更新一次,减少同步等待时间占整体时间的比重。该算法每隔n个iteration才同步更新一次,等效于增大batch size。但与之不同的是在n个iteration之间各个worker会根据本地的参数更新。\\
\begin{figure}[htp]
\centering
\includegraphics[width=10cm]{a_bsp}
\caption{bsp,a-bsp,ssp示意图}
\end{figure}
2015年,Sixin Zhang等[引用]提出异步的弹性均值sgd算法。其算法流程如图2.17所示。设计思想为:每个worker进行参数更新时,除了要考虑当前梯度,还应考虑当前worker的参数与中心节点master的全局参数的差距。不仅要减去当前worker的梯度,还应该减去这个差值。
\begin{figure}[htp]
\centering
\includegraphics[width=10cm]{easgd}
\caption{EASGD算法流程图}
\end{figure}
随后,2016年Peter H等[引用]基于EASGD算法提出gossiping算法。与EASGD算法相似,是一个去中心化等EASGD算法,其将中心节点更新的全局参数直接用各worker参数的均值代替了。该论文实验结果有诸多启发性的结论:1. 在小节点情况下(节点数在32个以内),异步的EASGD,gossiping SGD收敛速度要快于同步SGD;2. 当节点数增大(到100节点),同步SGD的可扩展性更好;3. 当更新步长(即lr)较小时,同步SGD的收敛效果要好,即异步更新下,步长不宜太小;4. gossiping SGD单次迭代时间小于同步SGD,在训练刚开始阶段work的很好,但当步长逐渐变小时,gossiping SGD的收敛速度变得很慢。\\
同时,Ioannis Mitliagkas等[引用]也在2016年提出一种通过调整momentum的值抵消异步更新梯度与全局实际梯度的差距。其设计思想是在异步SGD的情况下,因为每个worker求得的梯度并不是当前系统中最新权重的梯度,是相对较老权重的梯度,最新的权重梯度和计算出较老权重的梯度之间的差距实际上是一种隐含的动量。所以当使用异步SGD时,可以适当减小优化器中的momentum,因为异步SGD情况下本身就带有部分momentum,异步程度越大,即worker节点数越多,这种隐含的momentum就越大,此时优化器的momentum就应该越小。甚至优化器的momentum可以为负数,来抵消隐含momentum过大的影响。\\
该论文对异步更新方法提供了一种比较新奇的理解方式,但实际上关注点和EASGD,gossiping SGD是一致的,都是关注较老的参数与最新参数之间的差值。只不过EASGD,gossiping SGD是在worker的参数时,除了减去梯度,还减去了较老的参数与最新参数的差值。而不是仅仅减去梯度。而本文虽然仅仅减去梯度,但通过调整momentum来弥补该梯度的不准确性。\\
由于在大部分神经网络中异步更新算法无法满足程序精度要求,实际应用中往往使用严格同步的更新算法保证训练精度。下面是同步更新算法的相关研究进展。\\
2017年,Hao Zhang等[引用]结合深度学习训练特点,针对严格同步优化算法进行了系统优化,极大提高了分布式训练效率。论文提出两个解决分布式训练中网络传输瓶颈的方法:1. wait-free backpropagation(WFBP),将反向传播和参数同步嵌套在一起,将大部分网络通信开销隐藏在反向求导的计算中,从而极大减小同步时间;2. SFB方法:在参数量很大的网络中,某些全连接层的参数量很大,会造成网络瓶颈,为了减少这部分网络传输参数,提出将全连接层的梯度(M*N)进行矩阵分解成(M+N)个参数,进行广播传输,然后再通过矩阵乘法将(M+N)个参数恢复成(M*N)的参数。通过上面两种优化方法,使得系统近乎达到理想的线性加速比,相对于原生框架的分布式效率有极大提升,如图2.18所示。\\
\begin{figure}[htp]
\centering
\includegraphics[width=10cm]{poseidon_result}
\caption{Poseidon实验对比图}
\end{figure}
从图2.5对比TF与TF+WFBP可知,WFBP算法对性能提升有巨大帮助,在参数量相对较小的网络(如inception-V3)中该算法即可保证网络达到近线性加速比。而在网络参数量较到,特别是全连接层参数量较大时,单使用WFBP算法不足以压缩同步参数的时间开销。因为全连接层计算量小而参数量大,使得其对无法通过WFBP算法将通信延时隐藏在全连接层的反向计算中。而SFB算法正好解决了该问题。通过将全连接层M*N个参数分解成M+N个参数,极大减少了参数传输量。所以在全连接层占比较大的网络中,如VGG19,VGG19-22k等,加上SFB算法后分布式效率能接近理想加速比。\\
同年,Jianmin Chen等[引用]提出通过提供少量计算节点备份以缓解严格同步形成的木桶效应。论文首先通过大量实验验证了异步算法在神经网络训练中不能保证模型精度的问题。所以为保证训练精度应使用严格同步的优化算法。但是同步就导致每次迭代时间受计算最慢的机器的影响,随着集群增大,影响会越来越明显,为了避免这个影响,论文提出使用少量备份节点来减缓最慢机器导致的影响。假设集群中有53个计算节点同时计算,当server端接收到50个梯度后,就同步更新,剩下最慢的3个终止次计算,这样能有效减少最慢的时间等待。这种通过备份节点来缓解木桶效应带来的影响具有非常好的启发意义。同时论文也提出下一步工作:用time-out,代替backup节点的想法要跟进。当80\%的梯度到达时,就更新参数,以这80\%的梯度代替所有的梯度进行更新。\\
同时,facebook等[引用]在Training ImageNet in 1 hour中提出两个简单且极为有效的调整学习率的方法:线性增大学习率和热启动方法。首先论文以保证sgd算法意义出发,从公式中推导出当增大batch size时,线性增大学习率情况下,公式与原始batch size下的含义相同。同时在实际训练过程中,因为刚开始网络状态受初始化影响较大,且训练初期网络状态波动很大,学习率过大时,容易造成网络不收敛,故在训练刚开始阶段提出使用热启动的训练方法,从一个较小的学习率逐渐增长至线性增大策略所要求的学习率。实验结果如图2.19所示。\\
\begin{figure}[htp]
\centering
\includegraphics[width=10cm]{training1hour_result}
\caption{线性增大学习率和热启动方法下不同batch size下模型精度曲线}
\end{figure}
紧接着,伯克利尤洋等[引用]提出自适应学习率算法。论文思想是不同网络层之间参数数值差距较大,而对应的梯度差距也更大,往往存在数量级之间的差距。如图2.20所示AlexNet在第一次迭代中不同网络层参数与其梯度的比值可知,不同网络层之间参数与梯度比值差距非常大,如果以相同的学习率来更新参数,必定导致较小参数更新步长过大,导致网络不收敛。故论文提出自适应各层参数大小的学习率调整方法,进一步将batch size由8k提升至32k,在24分钟内训完ImageNet。\\
\begin{figure}[htp]
\centering
\includegraphics[width=10cm]{lars_grad}
\caption{AlexNet在第一次迭代中不同网络层参数与梯度的比值}
\end{figure}
随后,2018年Google Brain等[引用]提出通过逐渐增大batch size替换学习率衰减的方法,最终将batch size提升至64k,只用2500多次迭代就将ImageNet训到了理想精度。\\
\subsection{分布式深度学习中数据压缩的发展}
由2.1部分可知,分布式深度学习系统中通信量越小,分布式效率越高。近年来业界对在保证精度的情况下,如何尽可能地压缩传输梯度进行了相关研究。2017年,Xiangru Lian等[引用]提出去中心化的EASGD算法,减小网络通信量。在其环形拓扑结构中,每个worker只与其相邻的两个节点通信,将相邻两个节点的参数加和求平均,并求得本地参数与求得平均参数的差值更新至server。因为其更新的是参数的差值而不是参数本身,以此达到梯度压缩的目的。\\
2018年,Yujun Lin等[引用]提出稀疏化梯度的方法。即每次只更新前0.1\%-1\%的梯度,极大减少了网络带宽,压缩比能达到600倍。剩下99.9\%未更新的梯度保留在本地,随着迭代的进行依次累加,直到该梯度达到前0.1\%~1\%后更新。论文解释这种将梯度保留在本地的方法,类似于增大batch size。提出动量修正,梯度修剪,动量因子隐蔽,热启动等方法保证压缩后梯度的精度和延迟更新梯度的影响。其中动量修正示意图如图2.21所示。
\begin{figure}[htp]
\centering
\includegraphics[width=10cm]{dgc_picture}
\caption{动量修正原理示意图}
\end{figure}

\section{本章小结}
本章主要介绍了分布式训练神经网络的特点,数据并行和模型并行这两种并行方式;介绍了神经网络在图像分类、目标检测、自然语言处理等领域的发展;最后详细介绍了分布式深度学习中涉及到的模型,系统,优化算法相关的问题。





\chapter{低精度分布式更新算法}
本章主要介绍低精度分布式更新(LPDU)算法的实现原理,并详细介绍整个框架工作流程和性能优化的相关方法。通过分析不同应用的理论效率和实际效率结果,评估系统的性能。
\section{引言}
随着人工智能的快速发展,为提升训练效率,满足生产需求,业界已针对分布式深度学习系统进行了深入的研究。本章提出的LPDU算法希望通过低精度数据通信方法来减少分布式同步数据过程中的数据传输量,从而减小同步时间开销,提高分布式训练效率;同时,为保证神经网络的训练精度,使用混合精度更新方法对网络进行更新。本章将介绍如何在现有高效的分布式通信框架horovod上集成目前主流的深度学习框架,并进一步实现LPDU算法,分析低精度数据通信相对于原始浮点数通信的缺点,以及对应的改进方法;并介绍相关的优化方法,最后通过理论分析和实验结果比较,说明LPDU算法的有效性和实用性。
\section{MXNet与Horovod的整合}
horovod是uber针对tensorflow在分布式深度学习训练中的不足,基于百度提出的ring-allreduce构建。有3个主要特征:1. horovod与深度学习框架相互独立,用户以python包的形式调用horovod的API。使其发展独立于任何深度学习框架,目前horovod已经支持tensorflow和pytorch。这样tensorflow和pytorch用户可以在任意版本的深度学习框架中使用horovod,不需要担心兼容问题。目前MXNet团队已经将MXNet整合进了horovod,正在等待horovod团队整合。2. 用NCCL取代百度ring-allreduce实现。NCCL是NVIDIA的通信库,提供高度优化的ring-allreduce实现。通过NCCL的ring-allreduce实现,使得horovod的性能有巨大提升。3. 提出tensor fusion方法,通过将多个小的tensor整合成数据量稍大的tensor,进而对这个大的tensor做allreduce。提高了数据的通信效率。

\begin{figure}[htp]
\centering
\includegraphics[width=12cm]{horovod_mxnet_integration}
\caption{MXNet与horovod整合示意图}
\label{fig:horovod_mxnet_integration}
\end{figure}
horovod中主要涉及两个操作:allreduce和broadcast。其与MXNet的整合示意图如图~\ref{fig:horovod_mxnet_integration}所示。由图可知,horovod通过继承MXNet的优化器optimizer实现与MXNet的整合,即更新模型时使用horovod的DistributedOptimizer更新。对于MXNet而言,分布式程序只是一个使用DistributedOptimizer优化器更新模型的单机程序。在horovod中,因为其在真正更新之前,会对参数做allreduce来同步全局参数。这种整合方式使得MXNet与horovod可以各自独立发展,而不会产生版本的相互依赖。为保证各个节点中初始化模型的一致性, 其提供了一个broadcast parameters的API用于同步模型初始化的值。

各个深度学习框架通过继承horovod对外提供的tensor接口,提供自身的实现即可整合到horovod中。接口如下所示:

\begin{lstlisting}[language=C, numbers=none]
Tensor {
  dtype();  //  return the data type of tensor;
  shape();  //  return the shape of tensor;
  data();   //  return the data pointer of tensor;
  size();   //  return the total bytes of data;
}
\end{lstlisting}

由上伪代码可知,将深度学习框架整合进horovod只需通过继承tensor基类并实现对应的方法即可。而horovod只需通过调用tensor中的对应方法即可完成数据的同步。故为将MXNet整合入horovod,其设计如下:

\begin{lstlisting}[language=C, numbers=none]
MXTensor : Tensor{
public:
  MXTensor(NDArray* tensor);
  // override the 4 APIs
  dtype();
  shape();
  data();
  size();
protected:
  NDArray* tensor_;  // the pointer address to the data from MXNet
}
\end{lstlisting}

\section{LPDU算法的设计与实现}
\begin{figure}[htp]
\centering
\includegraphics[width=13cm]{ps_allreduce}
\caption{PS(a)和MPI(b)下数据并行的实现原理}
\label{fig:ps_allreduce}
\end{figure}

在数学形式上,给定数据集$D$,损失函数$L$,拟合参数为$\theta$的神经网络的过程,可以被建模成公式~\ref{equ:nn_equ}形式。其中$t$表示迭代次数,$\nabla L$表示在当前参数$\theta^{(t-1)}$,数据为$D^{(t)}\subset D$,学习率为$\epsilon$时,损失函数的梯度。在不断迭代过程中,直至$\theta$达到终止要求。
\begin{equation}
\label{equ:nn_equ}
\theta^{(t)}=\theta^{(t-1)}+\varepsilon\cdot\nabla L(\theta^{(t-1)},D^{(t)})
\end{equation}

在分布式深度学习中,通常采用数据并行的方式进行分布式训练,其通过将数据集$D$划分,放在不同计算节点(下标为$p=1,...,P$)上,如图~\ref{fig:ps_allreduce}所示。在每次迭代$t$中,每个计算节点从各自数据集$D_{p}$获取单次迭代数据$D^{(t)}_{p}$,并计算模型梯度$\nabla L(\theta^{(t)},D^{(t)}_{p})$;随后系统将各个节点梯度进行聚合,使用下列公式~\ref{equ:nn_dl_equ}对模型进行更新。
\begin{equation}
\label{equ:nn_dl_equ}
\theta^{(t+1)}=\theta^{(t)}+\varepsilon\sum^{P}_{p=1}\nabla L(\theta^{(t)},D^{(t)}_{p})
\end{equation}

在数据并行模式下,每个计算节点都要在本地维护一份全局共享参数$\theta$,这将产生大量的通信开销。目前主要有参数服务器(图~\ref{fig:ps_allreduce}a)和MPI(图~\ref{fig:ps_allreduce}b)两种方法来实现全局参数的维护。本章算法则是基于MPI通信模式提出。假设神经网络模型参数$\theta$的数据量为$M$,由图~\ref{fig:ps_allreduce}b可知,在每次迭代$t$中,每个worker会发送、接受模型当前梯度$\nabla\theta$,其大小与模型参数$\theta$相同,即每个worker都会产生$2M$的通信数据量。该通信数据量大小对分布式训练神经网络的效率影响至关重要。本章LPDU算法针对神经网络自身的容错性及其对数据精度不敏感的特点,将32比特浮点数表示的梯度数据$\nabla\theta$转换成16比特的bfloat16数据格式,使得$\nabla\theta$的数据量由$M$减少至$M/2$。此时,每次迭代$t$中每个worker传输的数据量从$2M$减少至$M$,使得分布式系统的通信开销降低,提升系统训练效率。且神经网络参数$\theta$的数据量越大,提升效果越明显。

LPDU算法主要包含两部分:低精度数据通信和混合精度更新。低精度数据通信部分旨在减少分布式训练过程中的同步通信开销,从而提升分布式训练效率;混合精度更新部分则是为避免低精度数据表示带来的数据精度损失造成神经网络训练精度下降的问题,将同步后的低精度梯度数据转换为原始浮点数,再对网络进行更新,可减小更新过程中的数据损失。使用该方法可保证在低精度数据通信情况下神经网络训练精度达到原始浮点数训练的相同精度。
\subsection{低精度数据通信算法}
%分析在训练过程中梯度的分布范围。确定在该范围内,浮点数与bf16数据的精度差距。

由2.1部分可知,相对于单机训练,分布式训练多了额外的同步通信开销,其对分布式训练效率的高低有至关重要的影响。结合业界对低精度数据在神经网络中的研究进展,本节提出一种高效的低精度数据通信方法,在混合精度更新算法配合下,能在不损失神经网络训练精度的情况下,提升分布式训练效率。

算法核心思路是:使用低精度数据进行通信,相对于浮点数通信减少了一半数据量,从而提升训练效率。根据业界对低精度数据在神经网络中的研究进展可知,目前主要有两种低精度数据格式用于神经网络训练,分别是半精度浮点数(FP16)和bfloat16(BF16)数据格式。这两种数据格式在表示数值范围和精度上均有不同程度的损失,通过对应的改进方法,可保证低精度数据下神经网络的训练精度达到原始浮点数训练相同精度。

考虑到FP16与BF16数据格式的特点,BF16数据格式与浮点数之间的转换开销较小;且其可表示的数值范围与浮点数基本相同,故不需要考虑FP16表示情况下梯度数据超过数值表示范围的问题。基于以上两种原因,LPDU算法采用BF16格式数据进行通信,算法内容如算法~\ref{alg:lpdc_alg}所示。下面介绍基于BF16格式的低精度数据通信算法的具体实现。

\begin{algorithm}\small
\caption{低精度数据通信算法LPDC}
\textbf{输入:}
浮点数数据:$Data_{fp32}$ \\
\textbf{输出:} 
全局同步后的低精度数据:$Data_{bf16}$
\begin{algorithmic}[1]
	\STATE{将原始32比特的浮点数$Data_{fp32}$转换为BF16格式的低精度数据$Data_{bf16}$}
    \STATE{使用$allreduce$对$Data_{bf16}$做全局同步}
    \STATE{返回全局同步后的低精度数据$Data_{bf16}$}
\end{algorithmic}
	\label{alg:lpdc_alg}
\end{algorithm}

基于BF16格式的低精度数据通信算法的实现主要包含两部分:1.在进行同步之前将原始浮点数转换成BF16格式数据;2.对BF16格式的梯度数据进行同步,求得全局梯度。本文在horovod基础上对低精度数据通信算法进行实现,具体工作包括:1.继承horovod提供的Tensor类,设计BF16 Tensor,供horovod调用,其需要完成原始浮点数到BF16格式数据的转换工作;2.扩展MPI,使其支持BF16格式数据同步,实现一个自定义的BF16求和函数,供MPI在做allreduce过程中对BF16数据求和使用。

针对以上两部分内容,具体设计如下所示。在MXTensor基础上,添加了新的BF16指针,用于存放BF16数据。在构造函数中,将深度学习框架传入的FP32的数据转换成BF16数据。针对第二点BF16的求和函数的实现。因为目前没有BF16的硬件计算单元,求和计算只能在浮点计算单元上计算。由下BF16 sum函数可知,函数主要分为三部分:1.将BF16数据转换为浮点数;2.对浮点数据进行求和;3.将求和后的浮点数结果转换为BF16数据。

\begin{lstlisting}[language=C, numbers=none]
// 1. bf16 tensor defination
MXBF16Tensor: MXTensor {
public:
  MXBF16Tensor(NDArray* tensor):MXTensor(tensor){
  // according the count of tensor elements allocate memory to store bf16 data;
  // convert source data of tensor to bf16 data format
}
  source_data();   // return the source pointer of NDArray;
  // override the 3 APIs as below:
  dtype();  // return bf16 flag;
  data();    // return data pointer of bf16;
  size();     // return the total bytes of bf16 data;
  private:
  unsigned short* bf16dptr_;
}

// 2. implement bf16 sum according MPI_op define
void bf16_sum(void* invec, void* inoutvec, int* len, MPI_Datatype* datatype) {
  for(int i = 0; i < *len; i++) {
    float tmp_in = convert_bf16_to_fp32(invec[i]);
    float tmp_out = convert_bf16_to_fp32(inoutvec[i]);
    tmp_out += tmp_in;
    inoutvec[i] = convert_fp32_to_bf16(tmp_out);
  }
}
\end{lstlisting}


\subsection{混合精度更新算法}
由表1.2可知,BF16格式数据有效数字仅有2~3位,相对于FP32格式数据存在一定精度损失。由随机梯度下降(SGD)更新算法公式~\ref{equ:sgd}可知,SGD将当前梯度乘以学习率$\eta$,再将值更新到模型参数$W$中。通常情况下$\eta$是一个小于1的数,并随着训练的进行,学习率$\eta$逐渐减少。若将$\eta*g_{t,i}$存放在BF16数据格式中,该过程将造成一定的精度损失,产生计算误差,最终该误差将传播到模型参数中。
\begin{equation}
\label{equ:sgd}
W_{t+1,i}=W_{t,i}-\eta*g_{t,i}
\end{equation}

为避免更新过程中产生的精度损失,本文借鉴FP16训练神经网络的方法,使用混合精度更新算法来对模型进行更新。如算法~\ref{alg:mpu_alg}所示。其核心思想是:在更新之前将BF16格式梯度数据转换成浮点数据格式,再使用浮点数梯度对网络模型进行更新,从而避免更新过程中产生的精度损失。混合精度更新算法下,神经网络训练流程如图~\ref{fig:MPUflow}所示。可知混合精度更新算法主要分为两部分:全局同步后的BF16格式梯度到浮点数的转换和SGD算法更新。

\begin{algorithm}\small
\caption{混合精度更新算法MPU}
\textbf{输入:}
本地网络层浮点参数:$Weight_{fp32}$,全局同步后的低精度梯度:$gradient_{bf16}$ \\
\textbf{输出:} 
更新后的网络层浮点参数:$Weight_{fp32}$
\begin{algorithmic}[1]
    \STATE{将全局同步后的低精度梯度$gradient_{bf16}$转换为32比特的浮点数梯度$gradient_{fp32}$}
    \STATE{使用更新算法(如随机梯度下降算法),把转换后的浮点梯度$gradient_{fp32}$更新到网络层参数$Weight_{fp32}$中}
    \STATE{返回更新后的参数$Weight_{fp32}$用于下一次迭代训练}
\end{algorithmic}
	\label{alg:mpu_alg}
\end{algorithm}

\begin{figure}[htp]
\centering
\includegraphics[width=13cm]{MPUflow}
\caption{混合精度更新算法下训练流程图}
\label{fig:MPUflow}
\end{figure}


同理,混合精度更新算法也适用于带动量的SGD,Adagrad,RMSprop等。实验证明混合精度更新算法能够保证在低精度数据通信方式下,神经网络的训练精度达到原始浮点数训练相同精度。

\subsection{LPDU算法设计}
由3.2部分可知,horovod通过继承深度学习框架的优化器实现自定义的分布式优化器DistributedOptimizer达到同步更新的目的。其中DistributedOptimizer的分布式更新算法如算法~\ref{alg:fp32_update}所示。为同步全局梯度,其在调用更新算法之前使用allreduce同步全局梯度,再用全局同步后的梯度更新本地参数,更新后的参数即全局参数。

\begin{algorithm}\small
\caption{原始分布式更新算法}
\textbf{输入:}
本地网络层参数:$Weight$,梯度:$gradient$ \\
\textbf{输出:} 
全局更新后的网络层参数:$Weight$
\begin{algorithmic}[1]
    \STATE{使用$allreduce$对网络层梯度$gradient$做全局同步}
    \STATE{使用更新算法(如随机梯度下降算法),把全局同步后的$gradient$更新到网络层参数$Weight$中}
    \STATE{返回全局更新后的参数$Weight$用于下一次迭代训练}
\end{algorithmic}
	\label{alg:fp32_update}
\end{algorithm}

由3.3.1和3.3.2部分可知,低精度分布式更新算法主要包含两部分:低精度数据通信(LPDC)和混合精度更新(MPU)算法。LPDC算法负责完成BF16格式梯度数据的同步,以减少分布式训练过程中的同步通信量,从而减少同步时间开销,提升分布式训练效率。MPU算法则是为了避免在更新阶段进一步造成精度损失,提出将BF16格式梯度数据转换为浮点数再进行更新的策略。在MPU算法下,使用低精度梯度数据训练神经网络的收敛精度能达到原始浮点数训练相同收敛精度。基于BF16格式数据的LPDU算法如算法~\ref{alg:bf16_update}所示。

\begin{algorithm}\small
\caption{低精度分布式更新算法LPDU}
\textbf{输入:}
本地网络层参数:$Weight$,梯度:$gradient$ \\
\textbf{输出:} 
全局更新后的网络层参数:$Weight$
\begin{algorithmic}[1]
	\STATE{使用LPDC算法求得全局同步后的梯度$gradient_{bf16}$}
    \STATE{使用MPU算法将$gradient_{bf16}$更新到网络层参数$Weight$中}
    \STATE{返回全局更新后的参数$Weight$用于下一次迭代训练}
\end{algorithmic}
	\label{alg:bf16_update}
\end{algorithm}

首先使用LPDC算法完成全局梯度数据的同步,其中通过将原始浮点数据转换为BF16格式数据,在损失部分数据精度情况下减少同步时间开销,以提升分布式训练效率;为避免全局同步后的低精度梯度数据带来的精度损失在更新过程中进一步传播,本文使用MPU算法对模型参数进行更新。LPDU算法下,分布式训练神经网络流程如图~\ref{fig:LPDUflow}所示。实验证明本文提出的低精度分布式更新算法,可在保证神经网络训练精度的前提下,减少分布式训练过程中的同步开销,提升神经网络的训练效率。相关实验结果将在3.4部分详细分析介绍。

\begin{figure}[htp]
\centering
\includegraphics[width=13cm]{LPDUflow}
\caption{低精度分布式更新算法下训练流程图}
\label{fig:LPDUflow}
\end{figure}

\subsection{LPDU算法实现与优化}
由上面内容可知,LPDU算法主要包括两部分:低精度数据通信和混合精度更新算法。其中低精度数据通信算法中主要涉及两个具体实现:1.浮点数到BF16格式数据的转换;2.扩展MPI,使其支持BF16格式数据同步,需要实现一个BF16数据格式的求和函数,供MPI程序在对BF16数据做allreduce时使用。混合精度更新算法中,基于原始SGD算法的实现,只需在其之前实现BF16到FP32数据的转换即可。

综上分析,基于horovod实现LPDU算法,需要完成3部分工作。其中LPDC算法部分工作在3.3.1部分已经介绍。MPU算法部分低精度数据到浮点数到转换部分本文通过在MXNet中的回调函数中实现,实现逻辑如下伪代码所示。在进行低精度数据同步后,程序将自动调用回调函数将低精度梯度转换成浮点数。最后返回给分布式优化器DistributedOptimizer用于模型更新。
\begin{lstlisting}[language=C, numbers=none]
// convert bf16 data to fp32 data on callback function.
void callback() {
        // convert bf16_tensor to fp32, assign to output
        mxnet.tensor.dptr = BF16ToFloat(bf16_pointer, len);
        handle_manager.MarkDone(handle);
        handle_manager.ExecuteCallback(handle);
      }
\end{lstlisting}

为减少LPDU算法中额外产生的数据类型转换和BF16求和函数的计算开销,我们通过使用intrinsic指令直接操作数据完成数据转换和计算,达到SSE汇编代码相同的性能效果。本文针对不同硬件架构和运行时情况,进行了最优实现。考虑到部分数据地址并不符合AVX512指令地址对齐要求,故需要额外使用movdqu指令处理数据地址不对齐的情况。下面分别是在数据地址满足AVX512指令地址要求和不满足其地址要求情况下,浮点数据转换到BF16数据的具体实现。同理,BF16数据转换到浮点数和BF16 sum中的相关实现与之类似。

\begin{lstlisting}[language=C, numbers=none]
// 数据不对齐情况:使用movdqu指令处理
inline void convert_f32_to_b16(const void* src, void* dst)
{
  __m512i y = _mm512_bsrli_epi128(_mm512_loadu_si512(src), 2);
  _mm256_storeu_si256((__m256i*)(dst), _mm512_cvtepi32_epi16(y));
}

// 数据对齐情况:直接使用intrinsic指令计算
inline void convert_f32_to_b16(__m512i* src, __m256i* dst)
{
  __m512i y = _mm512_bsrli_epi128(*src, 2);
  *dst = _mm512_cvtepi32_epi16(y);
}
\end{lstlisting}

考虑到不同硬件对intrinsic指令的支持程度不同,本文分别针对LPDU算法中涉及的每种操作进行了三种实现:Naive,AVX256,AVX512,以兼容不同硬件。Naive表示不使用任何硬件指令,C代码的实现;AVX256表示使用AVX256相关的硬件指令的实现;同理,AVX512则是使用AVX512硬件指令的实现。如表~\ref{tab:intrinsic_perf}所示。可知相对于Naive实现方法,AVX512实现有2.2~5.7倍的性能提升;AVX256实现有1.6~3.4倍的性能提升。

\begin{longtable}[c]{c*{6}{l}}
\caption{不同实现下性能加速比}\label{tab:intrinsic_perf}\\
\toprule[1.5pt]
 功能函数 & 数组大小 & 
\multicolumn{1}{c}{Naive实现} & \multicolumn{1}{c}{AVX512实现} &
\multicolumn{1}{c}{AVX256实现} & \multicolumn{1}{c}{AVX512} & 
\multicolumn{1}{c}{AVX256} 	\\
\multicolumn{1}{c}{} &\multicolumn{1}{c}{} &\multicolumn{1}{c}{(us)} & \multicolumn{1}{c}{(us)} &
\multicolumn{1}{c}{(us)} & \multicolumn{1}{c}{speedup} &
\multicolumn{1}{c}{speedup} 	\\

\midrule[1pt]%
\endfirsthead%

\multicolumn{7}{c}{续表~\thetable\hskip1em 不同实现下性能加速比}\\

\toprule[1.5pt]
 功能函数 & 数组大小 & 
\multicolumn{1}{c}{原始实现} & \multicolumn{1}{c}{AVX512实现} &
\multicolumn{1}{c}{AVX256实现} & \multicolumn{1}{c}{AVX512} & 
\multicolumn{1}{c}{AVX256} 	\\
\multicolumn{1}{c}{} &\multicolumn{1}{c}{} &\multicolumn{1}{c}{(us)} & \multicolumn{1}{c}{(us)} &
\multicolumn{1}{c}{(us)} & \multicolumn{1}{c}{speedup} &
\multicolumn{1}{c}{speedup} 	\\
\midrule[1pt]%
\endhead%
\hline%

\multicolumn{7}{r}{续下页}%

\endfoot%
\endlastfoot%
F32ToBF16 &	100 & 0.297 & 0.117 & 0.148 & 2.552 & 2.010  \\
F32ToBF16 &	1000 & 2.309 & 0.480 & 0.736 & 4.812 & 3.137  \\
F32ToBF16 &	10000 & 22.612 & 3.923 & 6.183 & 5.763 & 3.657  \\
F32ToBF16 &	100000 & 216.404 & 41.076 & 63.768 & 5.268 & 3.394 \\
F32ToBF16 &	1000000 & 2039.863 & 374.645 & 586.142 & 5.445 & 3.480 \\
BF16ToF32 &	100 & 0.279 & 0.109 & 0.169 & 2.549 & 1.649 \\
BF16ToF32 &	1000 & 2.117 & 0.442 & 0.986 & 4.794 & 2.147 \\
BF16ToF32 &	10000 & 20.492 & 3.795 & 9.129 & 5.400 & 2.245 \\
BF16ToF32 &	100000 & 267.662 & 103.113 & 158.061 & 2.596 & 1.693 \\
BF16ToF32 &	1000000 & 2715.713 & 1188.977 & 1658.944 & 2.284 & 1.637 \\
BF16 sum &	100 & 0.456 & 0.200 & 0.314 & 2.285 & 1.453 \\
BF16 sum &	1000 & 3.993 & 1.089 & 2.528 & 3.665 & 1.579 \\
BF16 sum &	10000 & 39.258 & 9.740 & 23.867 & 4.031 & 1.645 \\
BF16 sum &	100000 & 404.553 & 113.089 & 253.198 & 3.577 & 1.598 \\
BF16 sum &	1000000 & 3898.513 & 981.624 & 2390.526 & 3.971 & 1.631 \\
\bottomrule[1.5pt]
\end{longtable}

\section{实验与分析}
为说明低精度分布式更新算法的有效性,本节将从神经网络的训练性能和精度两方面进行对比分析。同时,为了说明LPDU算法的普适性,我们将分别展示图像分类、物体检测等领域网络在LPDU算法下的训练结果。为验证LPDU算法的高效性,本节将单独比较原始分布式更新算法与低精度分布式更新算法的效率。同时对实验细节和数据集也进行了详细介绍,使得实验结果和分析更具说服力。

\subsection{数据集介绍}
为说明本章提出的低精度分布式更新算法的广泛适用性。本章分别从典型的图像分类、物体检测应用中选择有代表性的神经网络在各自领域公开数据集上进行实验,分别与原始训练结果进行对比,验证低精度分布式更新算法的有效性。数据集信息如表~\ref{tab:datasets}所示。

ILSVRC2012: ImageNet数据集,包含1281167张图片,1000个类别,总容量约140GB,是图像分类领域公认的权威数据集。

PASCAL VOC: 为图像识别和分类提供了一整套标准化数据。本文使用其2007与2012年发布的数据集,用于训练物体检测网络。包含16551个有效物体框,共20个类别,总容量约2.8GB。

\begin{table}[htbp]
\centering
\begin{minipage}[t]{0.9\linewidth}
\caption{数据集概况}
\label{tab:datasets}
\begin{tabularx}{\linewidth}{l X X X }
\toprule[1.5pt]
{\song 数据集名称} & {\song 适用场景} & {\song 样本数量} & {	\song 类别数量}\\
\midrule[1pt]
ILSVRC2012 & 图像分类 & 1281167 & 1000\\
VOC2007+2012 & 目标检测 & 16551 & 20\\
\bottomrule[1.5pt]
\end{tabularx}
\end{minipage}
\end{table}

\subsection{实验环境介绍}
本文使用的软件主要有:MXNet 1.3版本,OpenMPI 4.0以及开源框架horovod。本章LPDU算法是在ctcyang/horovod:mxnet feature fp16分支基础上进一步设计实现。所有实验结果均是在双sockets的Xeon Gold 6148处理器的集群中运行所得。因为目前各个框架CPU端计算性能均基于单socket进行优化。在每个socket上跑一个训练实例可使训练效率达到最佳。故本文在每个机器节点上跑两个训练实例,使得每个socket上跑一个训练实例,以充分发挥CPU计算性能。

\subsection{LPDU算法性能分析}
由3.3部分算法分析可知,LPDU算法开销包含四部分时间开销:通信时间、FP32数据转换BF16数据开销、对BF16数据求和开销、BF16数据转换为FP32数据开销。原始更新算法开销包含两部分时间开销:通信开销与FP32数据求和开销。在不同数据规模下,测得8节点16实例实验环境中原始更新算法与LPDU算法时间开销如下表~\ref{tab:fp32_bf16_update_time}所示。可知在数据量大于10,000时,LPDU算法相对于原始更新算法有加速。随着数据量的增大,加速效果越明显。当数据量增大到1,000,000时,LPDU算法加速比能达到1.35,可以预测数据量越大,通信开销占比越大,BF16更新算法加速比也将进一步增大。

\begin{table}[htbp]
  \centering
  \caption{不同数据规模下两种更新算法时间开销}
  \label{tab:fp32_bf16_update_time}
  \begin{minipage}[t]{0.8\textwidth} 
    \begin{tabularx}{\linewidth}{|l|X|X|X|}
      \hline
      数组大小  & origin alg$^{*}$(us) & LPDU alg$^{**}$(us) & Speedup\\
      \hline
100 & 6842.00 & 6332.42 & 1.08 \\
1000 & 5213.00 & 5216.59 & 1.00 \\
10000 & 6317.00 & 6204.70 & 1.02 \\
100000 & 7245.00 & 6081.76 & 1.19 \\
1000000 & 20668.00 & 15259.92 & 1.35 \\
      \hline
    \end{tabularx}\\[2pt]
    \footnotesize
    *:原始分布式更新算法时间开销:通信和浮点数求和时间\\
    **:低精度分布式更新算法时间开销:FP32ToBF16,BF16ToFP32,通信和BF16求和时间
  \end{minipage}
\end{table}

本节以50层的Resnet网络为例,从理论上分析该网络在8节点16训练实例环境中,LPDU算法相对于原始更新算法的理论性能收益。 由表~\ref{tab:resnet50_params}可知,仅有2.8\%的参数的数据规模小于100,000,97.2\%参数的规模均大于100,000,64.57\%参数的规模大于1,000,000。根据表~\ref{tab:fp32_bf16_update_time}数据可知:在50层的Resnet网络中,LPDU算法相对于原始更新算法的更新时间将有1.19~1.35倍的加速,甚至大于1.35倍的加速。

\begin{table}[htbp]
\centering
\begin{minipage}[t]{0.9\linewidth}
\caption{Resnet50各层参数大小分布}
\label{tab:resnet50_params}
\begin{tabularx}{\linewidth}{l X X }
\toprule[1.5pt]
{\song 参数规模} & {\song 参数个数} & {\song 占比(\%)}\\
\midrule[1pt]
<100000 & 713920 & 2.80 \\
100000-1000000 & 8323072 & 32.63 \\
>1000000 & 16466920 & 64.57 \\
\bottomrule[1.5pt]
\end{tabularx}
\end{minipage}
\end{table}

在50层Resnet网络的真实训练场景中,原始更新算法与LPDU算法在不同训练实例个数的更新时间如表~\ref{tab:fp32_bf16_real_update_time}所示。在8节点16实例情况下,LPDU算法相对于原始更新算法的更新时间有1.3倍的性能提升,在8节点情况下同步时间由原始的0.458s降低至0.352s,可节省30\%的更新时间。与表~\ref{tab:fp32_bf16_update_time}和上述分析1.19~1.35倍的性能提升相一致。两者相互验证,说明了算法实现的正确性和有效性。 

\begin{table}[htbp]
  \centering
  \caption{不同训练实例中两种算法更新时间}
  \label{tab:fp32_bf16_real_update_time}
  \begin{minipage}[t]{0.8\textwidth} 
    \begin{tabularx}{\linewidth}{|l|X|X|X|X|}
      \hline
      节点数 & 实例数 & origin alg(s) & LPDU alg(s) & Speedup\\
      \hline
1 & 1 & 0.015 & 0.041 & 0.366 \\
1 & 2 & 0.075 & 0.076 & 0.987 \\
2 & 4 & 0.374 & 0.299 & 1.251 \\
4 & 8 & 0.428 & 0.337 & 1.270 \\
8 & 16 & 0.458 & 0.352 & 1.301 \\
      \hline
    \end{tabularx}\\[2pt]
    \footnotesize
    说明:每个实例配置均为resnet50, batch size=128\\
  \end{minipage}
\end{table}

为进一步比较原始更新算法与LPDU算法在实际训练过程中的性能差异,本节通过采样得到不同训练实例下,实际训练中每次迭代时间如表~\ref{tab:fp32_bf16_real_iter_time}所示。可知在8节点16训练实例情况下,LPDU算法的单次迭代时间相对于原始更新算法减少了0.12s,其来自于BF16算法同步时间的降低。该时间与表~\ref{tab:fp32_bf16_real_update_time}中同步时间减少部分相一致。因为LPDU算法降低了同步时间开销,使得分布式训练效率相对于原始更新算法有所提升。如在8节点16训练实例情况下,LPDU算法使得分布式训练效率由84.05\%提升至87.5\%,提升了3.5个百分点。不可否认,现阶段CPU训练神经网络速度远慢于GPU训练,使得分布式训练中通信压力相对较小。若使用GPU训练,单次迭代中计算时间将大大减少,使得同步时间占比增大。此时LPDU算法对提升分布式训练效率的作用将更加明显。

\begin{table}[htbp]
  \centering
  \caption{不同训练实例中两种算法迭代时间}
  \label{tab:fp32_bf16_real_iter_time}
  \begin{minipage}[t]{0.9\textwidth} 
    \begin{tabularx}{\linewidth}{|l|X|X|X|X|X|}
      \hline
      节点数 & 实例数 & origin time & LPDU time & origin & LPDU \\
       &  & (s/iter) & (s/iter) & scaling & scaling\\
      \hline
1 & 1 & 2.8158 & 2.8271 & 100.00 & 100.00 \\
1 & 2 & 2.8685 & 2.9990 & 98.16 & 94.27 \\
2 & 4 & 3.0450 & 3.0387 & 92.47 & 93.04 \\
4 & 8 & 3.3332 & 3.1877 & 84.48 & 88.69 \\
8 & 16 & 3.3501 & 3.2310 & 84.05 & 87.50 \\
      \hline
    \end{tabularx}\\[2pt]
    \footnotesize
    说明:每个实例配置均为resnet50, batch size=128\\
  \end{minipage}
\end{table}

由上述分析可知,神经网络的参数量越大,受网络带宽限制,同步开销越大,LPDU算法对分布式性能提升作用越大。同时,该算法普遍适用于各种主流神经网络,如图像分类,物体检测和循环神经网络等。表~\ref{tab:ssd_vgg_scaling}分别展示了LPDU算法在物体检测网络SSD和分类网VGG中的性能提升。可知模型稍小的SSD网络中,LPDU算法可使分布式训练效率由85.01\%提升至89.84\%,相对于原始更新算法提升了4.83个百分点;在模型较大的VGG网络中该算法性能提升更加明显,相对于原始更新算法,在8节点16训练实例情况下提升了7.13\%。
\begin{table}[htbp]
  \centering
  \caption{原始更新算法与BF16更新算法在SSD与VGG网络中的性能加速}
  \label{tab:ssd_vgg_scaling}
  \begin{minipage}[t]{0.8\textwidth} 
    \begin{tabularx}{\linewidth}{|l|X|X|X|X|}
      \hline
      实例数 & SSD$^{*}$ FP32 & SSD BF16 & VGG$^{**}$ FP32 & VGG BF16 \\
       & scaling & scaling & scaling & scaling\\
      \hline
1 & 100.00 & 100.00 & 100.00 & 100.00 \\
2 & 96.88 & 97.81 & 88.78 & 92.17 \\
4 & 91.97 & 95.18 & 86.65 & 90.06 \\
8 & 88.44 & 92.27 & 83.48 & 89.44 \\
16 & 85.01 & 89.84 & 79.42 & 86.55 \\
      \hline
    \end{tabularx}\\[2pt]
    \footnotesize
    *:SSD的模型大小为101MB\\
    **:VGG的模型大小为528MB
  \end{minipage}
\end{table}

\subsection{LPDU算法精度比较}
为说明LPDU算法不会对神经网络精度造成影响,本节分别比较在相同配置下,原始更新算法和LPDU算法下神经网络最终的收敛精度。图~\ref{fig:resnet50_4node_acc}为4节点8实例情况下原始更新算法与LPDU算法下Resnet50的收敛曲线。可知在训练集与验证集上LPDU算法均与原始更新算法的精度曲线基本重合,最终精度也保持一致,说明LPDU算法在分类网中不会造成网络的精度损失。不同节点实例下原始更新算法与LPDU算法在ImageNet数据集上的最终结果如表~\ref{tab:resnet50_diff_node_acc}所示。可知相同节点数下,LPDU算法与原始更新算法下收敛精度差距不超过0.3\%,该细微误差属于网络训练过程中的正常波动。说明不同节点下LPDU算法均能保证神经网络的训练精度收敛到理想精度,验证了LPDU算法的有效性与正确性。

\begin{figure}[htp]
\centering
\includegraphics[width=13cm]{resnet50_4node_acc}
\caption{Resnet50中原始更新算法与LPDU算法收敛曲线}
\label{fig:resnet50_4node_acc}
\end{figure}

\begin{table}[htbp]
\centering
\begin{minipage}[t]{0.9\linewidth}
\caption{Resnet50在不同节点数下最终精度}
\label{tab:resnet50_diff_node_acc}
\begin{tabularx}{\linewidth}{l X X X }
\toprule[1.5pt]
{\song 数据集} & {\song 节点数} & {\song 原始更新算法(\%)} & {	\song LPDU算法(\%)}\\
\midrule[1pt]
ILSVRC2012 & 1 &  76.18 & 76.15\\
ILSVRC2012 & 4 & 76.20 & 75.98\\
ILSVRC2012 & 8 & 76.08 & 76.13\\
\bottomrule[1.5pt]
\end{tabularx}
\end{minipage}
\end{table}
为验证LPDU算法对神经网络的普遍适用性,本节以物体检测领域经典网络SSD为例,使用LPDU算法对其进行训练,通过比较原始更新算法与本文算法在验证集的收敛曲线和最终收敛结果验证LPDU算法的有效性。如图~\ref{fig:ssd_4node_acc}所示:在4节点8实例情况下,LPDU算法与原始更新算法在验证集上的mAP曲线基本重合,说明LPDU算法在物体检测网络中不会造成网络的精度损失。不同节点实例下原始更新算法与LPDU算法在VOC数据集上的最终结果如表~\ref{tab:ssd_diff_node_acc}所示,可知相同节点数下,LPDU算法与原始更新算法下验证集结果差距不超过0.4\%,在正常数据波动范围内。该误差来源于神经网络正常训练误差和BF16数据造成的部分数据损失。说明不同节点情况下LPDU算法均能保证神经网络的训练精度收敛到原始FP32更新算法相接近的精度,验证了LPDU算法的有效性与正确性。 

\begin{figure}[htp]
\centering
\includegraphics[width=10cm]{ssd_4node_acc}
\caption{SSD中原始更新算法与LPDU算法收敛曲线}
\label{fig:ssd_4node_acc}
\end{figure}


\begin{table}[htbp]
\centering
\begin{minipage}[t]{0.9\linewidth}
\caption{ssd在不同节点数下最终精度}
\label{tab:ssd_diff_node_acc}
\begin{tabularx}{\linewidth}{l X X X }
\toprule[1.5pt]
{\song 数据集} & {\song 节点数} & {\song 原始更新算法(\%)} & {	\song LPDU算法(\%)}\\
\midrule[1pt]
VOC2007+2012 & 1 & 64.22 & 64.18\\
VOC2007+2012 & 4 & 64.00 & 64.06\\
VOC2007+2012 & 8 & 64.04 & 63.64\\
\bottomrule[1.5pt]
\end{tabularx}
\end{minipage}
\end{table}

\section{本章小结}
本章主要提出了低精度分布式更新算法以减小分布式训练过程中同步时间的开销,提升分布式训练神经网络的效率,同时保证神经网络训练精度达到原始浮点数训练相同精度。本章首先介绍了将horovod整合进MXNet的设计原理,然后介绍了LPDU算法中的两个算法组成:低精度数据通信LPDC和混合精度更新MPU算法。并分别介绍了LPDC算法和MPU算法的原理和设计思想,并进一步在此基础上介绍了LPDU算法的设计和实现,以及LPDU算法实现过程中优化性能的相关方法。最后对LPDU算法中各个模块的时间开销进行了详细分析,验证本章算法的高效性。最后将该算法用于训练图像分类网络和物体检测网络,通过比较LPDU算法与原始更新算法的训练效率与神经网络最终的收敛精度,说明了本章算法在不影响神经网络训练精度情况下,使得分布式训练效率有明显提升。也说明了LPDU算法对神经网络的普遍适用性。

\chapter{极限精度梯度压缩方法}
考虑到梯度数据普遍较小,绝大部分数据绝对值在0~1之间的特点,如表~\ref{tab:resnet50_1iter_grad_fabs}所示。在Resnet50网络中,每次迭代需要更新155个tensor的梯度,表中展示了在第一次迭代中各个tensor最大绝对值所处的范围。可知仅考虑每个tensor中最大绝对值情况下,53.55\%的tensor中最大绝对值在0~1之间;41.29\%的tensor中最大绝对值在1~10之间;仅有5.16\%的tensor中最大绝对值在10~50之间。
\begin{table}[htb]
\centering
\noindent\begin{minipage}{0.65\textwidth}
\centering
\caption{Resnet50第一次迭代中各层梯度最大绝对值分布}
\label{tab:resnet50_1iter_grad_fabs}
\begin{tabular}{p{2.5cm}p{2.5cm}p{2.5cm}}
\toprule[1.5pt]
梯度绝对值 & 数量 & 占比(\%) \\\midrule[1pt]
0~1.0 & 83 & 53.55\\
1.0~10.0 & 64 & 41.29\\
10.0~50.0 & 8 & 5.16\\
\midrule[1pt]
\end{tabular}
\end{minipage}
\end{table}

随着训练的进行,网络逐渐趋向于稳定,梯度数据的波动范围更小,更集中于0~1之间。如表~\ref{tab:resnet50_21iter_grad_fabs}所示。在第21次迭代中绝对值在0~1之间的梯度占比为74.84\%,1~10和10~50之间的梯度数量相对减少。可知在后面训练的迭代中梯度数据的绝对值将更加集中于0~1之间。故本章希望针对梯度数据这一特点,使用更少比特位表示梯度。通过减少梯度数据的表示位,可减少分布式训练过程中的同步数据量,进而提高分布式系统的可扩展性和训练效率。
\begin{table}[htb]
\centering
\noindent\begin{minipage}{0.65\textwidth}
\centering
\caption{Resnet50第21次迭代中各层梯度最大绝对值分布}
\label{tab:resnet50_21iter_grad_fabs}
\begin{tabular}{p{2.5cm}p{2.5cm}p{2.5cm}}
\toprule[1.5pt]
梯度绝对值 & 数量 & 占比(\%) \\\midrule[1pt]
0~1.0 & 116 & 74.84\\
1.0~10.0 & 37 & 23.87\\
10.0~50.0 & 2 & 1.29\\
\midrule[1pt]
\end{tabular}
\end{minipage}
\end{table}

本章希望在不影响神经网络训练精度的情况下,尽可能减少梯度数据所需的比特位数,为进一步减少分布式通信开销,提高分布式系统可扩展性和训练效率提供数据压缩方法。本章将主要介绍梯度压缩的思路与实现方法,以及对应压缩方式在图像分类和物体检测网络中的训练精度。通过其与原始的训练精度对比,说明对应压缩算法的有效性和适用范围。本章针对图像分类、物体检测任务分别提出三种极限精度梯度压缩EPGC方法。图像分类任务对梯度数据精度要求低,本文提出9比特梯度压缩方法和8比特梯度压缩方法;考虑到物体检测网络对梯度数据精度要求偏高,具体原因将在后面详细分析,本文提出10比特梯度压缩方法。经实验证明:本文提出的三种极限精度梯度压缩EPGC方法,在图像分类或物体检测任务中均能达到原始训练相同收敛精度。说明了这三种梯度压缩方法在各自任务中的有效性,以及通过该压缩方法在保证神经网络收敛精度前提下,提升分布式系统可扩展性和训练效率的可行性。
\section{引言}
根据第二章内容可知,分布式训练过程中同步开销的大小对系统的可扩展性和训练效率有至关重要的影响。在不影响训练精度的情况下,将梯度尽可能地压缩则可减少同步开销,提高分布式系统的可扩展性和训练效率。本章将分别介绍适用于图像分类和物体检测网络的不同梯度压缩方法,将详细分析其具体实现和各个压缩方式下分类网和物体检测网络的最终收敛精度。

由第三章LPDU算法实现可知,基于horovod要实现新的数据压缩方法,则需在MPI中新增一个数据类型来解析该压缩数据,并实现一个该数据类型的求和函数供MPI调用。为快速验证本文针对图像分类和物体检测任务分别提出的梯度压缩方法对网络训练精度的影响,本章提出的梯度压缩方法均通过对FP32浮点数或半精度浮点数对应比特位清零的方式来模拟对应的压缩方法。该模拟方法可避免对MPI的改动,使得本章提出的压缩方法能够快速得到验证。下面将对适用于图像分类网络的两种EPGC方法:9比特梯度压缩方法和8比特梯度压缩方法;以及适用于物体检测网络的10比特梯度压缩方法进行详细介绍。
\section{分类网络中的梯度压缩方法}
本节主要介绍适用于图像分类网络的两种梯度压缩方法:9比特梯度压缩方法和8比特梯度压缩方法。实验证明这两种压缩方法在分类网络中均能达到理想收敛精度,但在性能上各有优缺点。

9比特梯度压缩方法核心思路是:在浮点数基础上,去除所有尾数比特位,仅保留1位符号位,8位指数位来表示梯度。在实现过程中浮点数与9比特数据格式的转换可以高效完成;而8比特梯度压缩方法核心思路是:将浮点数梯度转换为16比特的半精度数据,再在FP16数据基础上去除8位尾数位,仅使用8比特(1位符号位,5位指数位,2位尾数位)表示梯度。从其压缩思路可知,在实现过程中需要完成浮点数到半精度数据的转换和半精度数据到8比特数据的转换,其开销稍大于9比特梯度压缩方法。但该压缩方法所需传输的数据量少于90比特梯度压缩方法;其在半精度训练场景中,该方法能够更为高效实现。
\subsection{9比特梯度压缩方法}
根据上一章LPDU算法可知,BF16数据格式相对于原始FP32数据而言,去除了浮点数尾数中的后16比特位,仅保留7位有效比特位。本压缩算法根据这一思路,在BF16数据基础上,进一步减少尾数有效位,在保证训练精度情况下,尽可能减少尾数比特位,以减少分布式同步中的数据通信量。
\begin{figure}[htp]
\centering
\includegraphics[width=10cm]{simulate_11bits}
\caption{模拟11比特数据表示示意图}
\label{fig:simulate_11bits}
\end{figure}

本节通过对原始FP32浮点数尾数位清零的方法模拟这种压缩方法。如在FP32浮点数基础上仅保留两位有效尾数位,形成11比特的数据格式,可通过将FP32浮点数的低21位清零达到相同效果,如图~\ref{fig:simulate_11bits}所示。可知低21比特位清零后的浮点数等效于11比特位数据格式所表示数据。

使用这种模拟方法,本文逐渐减少尾数比特位,并使用对应梯度压缩方法训练图像分类领域经典网络Resnet50,来验证该压缩方法在图像分类任务中的有效性。最终发现在去除所有尾数情况下,仅使用9比特梯度压缩算法即可保证分类网的训练精度,故本节提出了适用于图像分类网络的9比特梯度压缩方法。

\subsection{9比特梯度压缩方法实验分析}
通过这种模拟方法,本文分别得出使用11比特,10比特,9比特情况下,压缩算法对分类网的影响,以Resnet50为例。经实验验证:在同步梯度数据时仅保留两比特位的尾数情况下,分类网也可收敛到原始浮点数梯度相同精度。在11比特有效位基础上,继续减少尾数位,保留1位尾数、去除所有尾数位情况下,在分类网上均能达到理想收敛精度。在去除所有尾数位仅使用9比特数据表示梯度情况下,Resnet50的收敛曲线如图~\ref{fig:simulate_9bits_acc}所示。

\begin{figure}[htp]
\centering
\includegraphics[width=13cm]{simulate_9bits_acc2}
\caption{9比特梯度压缩方法与原始浮点数表示下Resnet50训练精度曲线}
\label{fig:simulate_9bits_acc}
\end{figure}

保留2位尾数,1位尾数,不保留尾数情况下,Resnet50在验证集准确率如表~\ref{tab:simulate_11_10_9bits_acc}所示。所有结果均在4节点8实例配置下训练得到。由表~\ref{tab:simulate_11_10_9bits_acc}可知,在原始浮点梯度基础上,仅保留2位尾数,1位尾数,不保留尾数情况下,Resnet50网络精度与原始浮点数梯度精度相差<0.3\%。说明在误差允许范围内,在分类网中仅使用9比特数据(1个符号位,8个指数位)表示梯度即可满足分类网的训练精度要求。通过去除浮点数中所有尾数位的方法可在不影响神经网络训练精度前提下减少分布式同步开销,提升分布式系统的可扩展性和训练效率。

\begin{table}[htb]
\centering
\noindent\begin{minipage}{0.6\textwidth}
\centering
\caption{不同尾数位下Resnet50准确率}
\label{tab:simulate_11_10_9bits_acc}
\begin{tabular}{p{2cm}p{2.5cm}}
\toprule[1.5pt]
有效尾数位 & Resnet50 acc(\%) \\\midrule[1pt]
FP32 & 76.12 \\
2 & 76.08 \\
1 & 76.01 \\
0 & 75.87 \\
\midrule[1pt]
\end{tabular}
\end{minipage}
\end{table}

\subsection{8比特梯度压缩方法}
由本文第二章可知,半精度浮点数也可用于训练神经网络,也能保证网络的训练精度与原始浮点数训练一致。半精度浮点数与BF16数据格式相比,少了3个指数位,仅5个指数位。根据上一节9比特梯度压缩方法可知,在仅保留2位,1位尾数或不保留尾数位情况下,均能保证图像分类网络训练精度与原始浮点数训练精度在误差允许范围内。本节结合半精度浮点数特点,提出8比特梯度压缩方法:在半精度浮点数基础上,去除16位半精度浮点数中低8位尾数,仅保留两个有效尾数位,使用8比特数据表示梯度,以此将梯度压缩成单字节数据。由前文可知:该方法相对于9比特梯度压缩方法而言,在使用浮点数训练情况下其实现开销较大;但其通信开销较小,且在半精度浮点数训练模型下,8比特梯度压缩方法能够更高效的实现。

为快速验证本节所提出的梯度压缩方法的有效性,本节通过将半精度浮点数特定尾数位清零的方式模拟该压缩方法。原理如图~\ref{fig:simulate_8bits}所示。可知将FP16格式数据低8比特位清零后的半精度浮点数等效于8比特位数据格式所表示数据。

\begin{figure}[htp]
\centering
\includegraphics[width=10cm]{simulate_8bits}
\caption{模拟8比特数据表示示意图}
\label{fig:simulate_8bits}
\end{figure}

本文在Resnet50训练中使用这种8比特梯度压缩方法,通过比较网络与原始浮点数训练的收敛精度来验证该梯度压缩方法的有效性和适用范围。实验发现本节提出的8比特梯度压缩方法在分类网训练中不会对网络精度造成影响。

在该8比特梯度压缩方法下,神经网络训练流程如图~\ref{fig:8bits_workflow}所示。整个训练流程类似于第三章LPDU算法下的训练流程。但8比特梯度压缩方法的数据转换过程较为复杂。首先需要将浮点数转换成半精度浮点数,该过程涉及到数据位的转换,比较耗时;再在半精度浮点数基础上去除8比特尾数位,仅使用8比特梯度数据进行通信。完成8比特的数据通信后,通过8比特数据到半精度浮点数,半精度浮点数到浮点数的转化,将8比特数据恢复成浮点数。当使用半精度浮点数训练神经网络时,8比特梯度压缩方法的训练流程则可直接去除浮点数与半精度数据的转化过程,在这种情况下8比特梯度压缩方法最为高效。
\begin{figure}[htp]
\centering
\includegraphics[width=10cm]{8bits_workflow}
\caption{8比特梯度压缩方法下训练流程图}
\label{fig:8bits_workflow}
\end{figure}

\subsection{8比特梯度压缩方法实验分析}
在4节点8实例情况下,使用该方法将梯度数据压缩成单字节数据,Resnet50的收敛曲线如图~\ref{fig:simulate_8bits_acc}所示。可知本节提出的8比特梯度压缩方法同样能达到原始训练精度:76.1\%,说明本节提出的压缩方法可在不影响分类网络精度情况下,减小分布式训练神经网络中的通信量,从而提升分布式系统的可扩展性和训练效率。
\begin{figure}[htp]
\centering
\includegraphics[width=13cm]{simulate_8bits_acc2}
\caption{8比特位梯度压缩方法与原始浮点数表示下Resnet50精度曲线}
\label{fig:simulate_8bits_acc}
\end{figure}

根据上一节提出的9比特压缩方法可知,可进一步在半精度浮点数上减少尾数位或去除所有尾数位,极限情况下仅使用6比特数据表示梯度。下一步将对这种可能的6比特梯度压缩方法在分类网络中进行尝试验证,通过比较该梯度压缩方法下分类网的最终收敛精度与原始收敛精度的差异,说明该梯度压缩算法对分类网精度的影响,从而确定该梯度压缩算法应用于分类网络的可行性。

\section{物体检测网络中的梯度压缩方法}

在目前一阶段、二阶段的物体检测网络中,均是把物体检测任务当成多任务来处理。主要包含:分类任务和回归任务。回归任务目的是通过逻辑回归的方式得到物体的坐标位置,即$(x,y,w,h)$, $x,y$表示物体框的中心坐标,$w,h$则表示物体框的宽和高。分类任务则是判断对应物体框的类别,与分类网类似。这两个任务共同完成物体检测任务。回归任务中关于$(x,y,w,h)$的计算如公式~\ref{equ:regression_equ}所示。

\begin{equation}
\label{equ:regression_equ}
\begin{split}
\hat{G_{x}}=P_{w}t_{x}+P_{x} \\
\hat{G_{y}}=P_{h}t_{y}+P_{y} \\
\hat{G_{w}}=P_{w}e^{t_{w}} \\
\hat{G_{h}}=P_{h}e^{t_{h}} 
\end{split}
\end{equation}


在SSD中使用该梯度压缩算法训练,其精度与原始网络训练精度差距较大,在保留2位尾数,1位尾数,去除所有尾数情况下,SSD的收敛曲线如图~\ref{fig:simulate_9bits_ssd_acc}所示。可知,在11比特压缩方法下,SSD的收敛精度已经与原始浮点数训练精度有一定差距。说明物体检测网络对梯度精度要求要高于图像分类网络。随着尾数位的不断减少,SSD的收敛精度也越来越低。
\begin{figure}[htp]
\centering
\includegraphics[width=12cm]{simulate_9bits_ssd_acc}
\caption{SSD在不同比特压缩算法下mAP曲线}
\label{fig:simulate_9bits_ssd_acc}
\end{figure}

保留2位尾数,1位尾数,不保留尾数情况下,Resnet50与SSD在验证集准确率如表~\ref{tab:simulate_11_10_9bits_acc}所示。所有结果均在4节点8实例配置下训练得到。由表~\ref{tab:simulate_11_10_9bits_acc}可知,在原始浮点梯度基础上,仅保留2位尾数,1位尾数,不保留尾数情况下,Resnet50网络精度与原始浮点数梯度精度相差<0.3\%。说明在误差允许范围内,在分类网中仅使用9比特数据(1个符号位,8个指数位)表示梯度即可满足训练精度要求。通过去除浮点数中所有尾数位的方法可在不影响神经网络训练精度前提下减少分布式同步开销,提升分布式系统的可扩展性和训练效率。而物体检测网络对梯度精度要求较高,本节提出的压缩方法则不适用于物体检测网络的训练。

\begin{table}[htb]
\centering
\noindent\begin{minipage}{0.6\textwidth}
\centering
\caption{不同尾数位下resnet50与SSD准确率}
\label{tab:simulate_11_10_9bits_acc}
\begin{tabular}{p{2cm}p{2.5cm}p{2.5cm}}
\toprule[1.5pt]
有效尾数位 & Resnet50 acc(\%) & SSD mAP(\%) \\\midrule[1pt]
FP32 & 76.12 & 64.00 \\
2 & 76.08 & 59.66 \\
1 & 76.01 & 55.33 \\
0 & 75.87 & 21.36 \\
\midrule[1pt]
\end{tabular}
\end{minipage}
\end{table}
为找出适用于物体检测网络的压缩方法,下一步将基于11比特梯度压缩方法,进一步增加尾数位,对物体检测网络进行分布式训练,直到找到在保证网络收敛精度的前提下,所需的最少比特位。



8bits

物体检测网络SSD在该压缩方法下mAP曲线如图~\ref{fig:simulate_8bits_ssd_acc}所示。可知SSD网络在本节8比特梯度压缩方法下最终mAP为59.78\%,与原始浮点数训练的结果:64.00\%差距较大。造成SSD训练精度的损失来自于尾数位精度的降低,其原因与9比特梯度压缩方法造成SSD训练精度下降的原因相同。说明本节提出的8比特梯度压缩方法在图像分类网络训练中不会对训练精度造成影响,可将该梯度压缩算法用于分布式训练分类网,以提升分布式系统的扩展性和训练效率。但因为物体检测网络对梯度精度要求较高,本节提出的8比特梯度压缩算法不能保证物体检测网络的训练精度,其不适用于物体检测网络。
\begin{figure}[htp]
\centering
\includegraphics[width=12cm]{simulate_8bits_ssd_acc}
\caption{8比特位梯度压缩方法与原始浮点数表示SSD mAP曲线}
\label{fig:simulate_8bits_ssd_acc}
\end{figure}

同时,根据上一节提出的9比特压缩方法可知,可进一步在半精度浮点数上减少尾数位或去除所有尾数位,极限情况下仅使用6比特数据表示梯度。下一步将对这种可能的6比特梯度压缩方法在分类网络中进行尝试验证,通过比较该梯度压缩方法下分类网的最终收敛精度与原始收敛精度的差异,说明该梯度压缩算法对分类网精度的影响,从而确定该梯度压缩算法应用于分类网络的可行性。

另一方面,对于本节8比特梯度压缩方法对物体检测网络造成精度下降的问题,下一步将在8比特压缩方法基础上,逐步增加尾数位,分别训练物体检测网络,直到物体检测网络训练精度达到原始浮点数训练精度为止。


\section{本章小结}
本章基于减少分布式训练神经网络通信开销,提高分布式系统的可扩展性和训练效率的目的,希望通过减少梯度数据比特位的方法来减少通信数据量,提出两种极限梯度压缩方法:9比特梯度压缩方法和8比特梯度压缩方法。经验证本章提出的两种梯度压缩方法在Resnet50网络中均能达到训练精度要求,说明这两种梯度压缩方法应用于分类网络的可行性;也可看出物体检测网络对梯度精度要求较高,本章提出的两种压缩算法不适用于物体检测网络。下一步主要有两方面工作:1.基于8比特梯度压缩方法,继续减少尾数位,极限情况下仅使用6比特数据传输梯度,将其应用于分类网的训练,通过比较其训练精度与原始训练精度的差异,验证这种梯度压缩算法的可行性;2.针对8比特梯度压缩方法在物体检测网络上造成精度损失的问题,基于8比特梯度压缩方法,逐渐增加尾数位,直至其在物体检测网络中的训练精度达到原始浮点数训练精度为止。








\chapter{总结与展望}
\section{工作总结}
优化分布式训练神经网络的性能对神经网络的发展有至关重要的作用,是所有神经网络算法的基础。随着神经网络的发展,数据量和神经网络规模呈现爆发性增长趋势,在可预见的将来单机训练已经很难满足要求,分布式训练神经网络将是必然选择,优化分布式训练神经网络的性能显得尤为重要。针对这一问题,本文提出BF16分布式更新算法,通过减少原始浮点梯度的尾数位,以减少同步梯度过程中的数据量,进而减少同步开销,提升分布式训练效率。在此基础上,本文提出9比特梯度压缩方法和8比特梯度压缩方法,以进一步减少同步时间开销。本文的主要研究工作和研究成果如下:

1. BF16分布式更新算法。该算法旨在保证神经网络训练精度的情况下,提升分布式训练神经网络的性能。通过使用BF16格式数据传输梯度,减少了梯度数据的尾数位,达到减少同步数据量,减少同步时间开销的目的。本算法受启发于google公司使用BF16数据格式训练神经网络的相关研究。其研究表明在神经网络训练过程中,完全使用BF16数据格式训练神经网络能达到浮点数训练相同的精度,并根据这一研究成果开发出来首款支持BF16计算硬件:TPU。故本文提出将原始浮点数梯度转换为BF16数据格式,以减少分布式同步过程中的通信量。

2. 9比特梯度压缩方法与8比特梯度压缩方法。在BF16更新算法基础上,为进一步压缩梯度数据比特位,本文提出基于原始浮点数去除所有尾数位的9比特压缩方法,仅保留1个符号位与8个指数位。为快速验证本文所提出的压缩方法对神经网络训练精度的影响,本文通过对浮点数梯度特定尾数位清零的方法模拟相应的压缩方法。根据近年来业界将半精度浮点数应用于神经网络训练中的相关工作,本文还提出基于半精度浮点数的8比特梯度压缩方法,保留1个符号位,5个指数位和2个尾数位。

在BF16更新算法相关实验中,本文首先通过分析BF16更新算法中涉及到的各部分操作的时间开销,以及特定神经网络的参数量大小与分布,从理论上分析BF16更新算法在特定神经网络训练中对分布式训练效率的性能提升。真实训练过程中测得的性能提升与理论分析相一致,说明算法实现与理论分析的正确性。而后通过比较BF16更新算法与原始更新算法在分类网,物体检测网的收敛效果与训练速度,说明BF16更新算法在不影响神经网络精度情况下,对分布式训练神经网络性能有一定提升。

在9比特梯度压缩方法与8比特梯度压缩方法相关实验中,本文通过对原始浮点数或半精度浮点数特定尾数位清零的方法模拟对应压缩方法。通过这两种压缩方法下神经网络的收敛精度与原始收敛精度的对比,说明本文提出的两种压缩方法的有效性。
\section{工作展望}
本文提出的9比特梯度压缩方法与8比特梯度压缩方法均通过在浮点数或半精度浮点数上模拟实现。现阶段只证明该压缩方法在分布式训练分类网中的可行性。对分布式训练效率的提升效果有待进一步实现这两种压缩算法后,在真实训练场景中的表现。下一步我们将基于horovod实现这两种压缩算法,进一步提升分布式训练效率。

为验证9比特梯度压缩算法与8比特梯度压缩算法并不仅仅适用于分类任务,我们将进一步在物体检测,自然语言处理任务中使用这两种压缩算法。根据不同任务中压缩算法对网络性能的影响,改进这两种压缩算法,找到一种普遍适用于神经网络的梯度压缩算法。


%%% Local Variables:
%%% mode: latex
%%% TeX-master: "../main"
%%% End:

\begin{ack}
  光阴荏苒、岁月如梭,转眼间在国防科学技术大学学习已经两年有余,美好的研究生生涯即将结束。在这两年多的时间里,很幸运有诸多老师,同学,朋友的陪伴。在学习上遇到困难时能及时给予我帮助,启发我进一步思考,为我提供了如此快乐,充实的学习生活;在生活上遇到问题时能及时为我开导,帮助我从消极情绪中快速走出来,让我一直以积极,乐观的心态面对各种困难。毕业在即,我要向所有给予我指导和帮助的老师、同学、朋友以及家人表示衷心的感谢!
  
  特别感谢我的指导老师李东升指导员。李老师学识渊博,和蔼可亲,在我还没正式步入研究生生涯时就给予了我很多帮助,解答我的疑惑,给我学习资料,安排师兄带我学习,使我尽早地进入了研究生生活,融入实验室大家庭。研究生期间李老师总是尽可能地满足我个人的研究兴趣并加以指导,感谢老师给我们提供了如此自由、良好的学习氛围。
  
  衷心感谢张钊宁老师。是张老师把我领进了深度学习的大门。在他的指导下,我做了许多有意义工作,也让自己的能力有了很大的提升。张老师也是我们的师兄,学习生活上有任何问题,困惑都可以从朋友的角度跟他反应交流。感谢张老师在我对毕设方向有疑惑动摇以及对未来工作担心时,及时和我沟通,帮我推荐工作。在张老师的指导、帮助下,使我的科研、动手能力有了很大提升,也让我对未来工作更加自信。
  
  衷心感谢intel公司赵鹏老师以及组内同事在我做毕设时给予我的帮助。在我对某一方面知识有疑问时,赵老师总是找相关领域的工作人员为我解答疑问;在我毕设遇到问题时,及时帮我分析问题,指明我的方向。感谢赵老师和Jason在我毕设和组内任务有冲突时,以我毕设为主,让我有足够的时间完成毕设。衷心感谢同事们在我有疑问时为我清晰解答相关问题,在你们的帮助下让我得以完成毕设内容,也让我对相关领域知识有了深入了解。
  
  感谢实验室其他指导老师:彭宇行老师,王意洁老师,张一鸣老师,沈思琪老师,你们的研究态度和对科研的追求,是我们学习的榜样。感谢实验室师兄师姐:尹璐伽,彭宝云,赵云祥,陈心圆,黎敏讷,陈圣灵,李真真等和众多同级的小伙伴:秦政,于昊,卢孟龙,左钟融,徐军,邵旭颖以及其他师弟师妹们在学习,生活中给予的帮助,你们的陪伴使得实验室的科研氛围充满了快乐。另外还要感谢PDL实验室的工作人员,你们对我学习之外的诸多事宜给予了充分的支持。

\end{ack}


\cleardoublepage
\phantomsection
\addcontentsline{toc}{chapter}{参考文献}
\bibliographystyle{bstutf8}
\bibliography{ref/myrefs}

\begin{resume}

  该论文作者在学期间取得的阶段性成果(学术论文等)已满足我校硕士学位评阅相关要求,为避免阶段性成果信息对专家评价学位论文本身造成干扰,特将论文作者对阶段性成果信息隐去。
  
\end{resume}

\end{document}
